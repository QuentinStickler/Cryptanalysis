\documentclass[
	aspectratio=169,	% Modern aspect ratio (TODO: Other ratios not yet supported)
	onlytextwidth,		% Sets totalwidth=\textwidth and therefore e.g. columns won't invade the margins
	t,					% Default vertical alignment of frames and colums at top (default is centered) % Stored in \beamer@centered (\beamer@centeredfalse, \beamer@centeredtrue)
%	handout,			% Create a basic handout of the presentation (removes overlays)
	]{beamer}

%%%%%%%%%%%%%%%%%%%%%%%%%%%%%%%%%%%%%%%%%%
% 1) Load the desired presentation theme
\usetheme[
% Individual options to customize the presentation conforming to the corporate design
	cb,					% Change default faculty color set and predefined faculty values: hs or <empty> (default), inw, cb, me, sw, wi, inst (\renewcommand{\insertfacultyname}{Institutname} needed)
	language=english,	% Change language to english (default: ngerman), other languages are possible (see babel-package) but may need further adjustments 
	toc,				% Adds a ToC slide
%	sectionslide,		% Display separate section slides
%	subsectionslide,		% Display separate subsection slides containing section and subsection name
%	smallpagenumber,	% Reduces the size of the page number
%	nototalpages,		% Hides the total number of slide in footline
%	nofacultyicon,		% Hides the faculty icon on title page
	colormath=nottext,	% Enables coloring of math text: off or <empty>, full, nottext (default)
%	printhandout,		% Places two slides on a single a4 paper for printing the presentation (beamer class option 'handout' needed)
%	noframesubtitle,	% Disables frame subtitles alltogether and slightly increases frame height
% Additional style options not completely conforming to the corporate design
	titlepagedate,		% Shows the date on the titelpage
%	fancystyle,			% Enables some fancy styles, that are not part of the corporate design specifications (default: off)
	progressbar,		% Shows the progressbar in footline (run twice to update progressbar)
	]{hsmw} 

%%%%%%%%%%%%%%%%%%%%%%%%%%%%%%%%%%%%%%%%%%
% 2) Specify default fields for presentation and pdf document properties
% Set the title: \title{Long title everywhere} or \title[Short title for footline]{Long title for titlepage}
\title[RC4-Algorithmus]{RC4 Algorithmus}
% Authors (separate multiple author names e.g. with \and for additional space): \author{author for everywhere} or \author[author for footline]{author for titlepage and thankyouslide}
\author[Quentin Stickler]{Quentin Stickler, B.Sc.}
% Institute (will be prefilled automatically, depending on chosen faculty theme option): \institute{institute for everywhere} or \institute[institute for footline and thankyouslide]{institute for titlepage}
%\institute[institute for footline and thankyouslide]{institute for titlepage}
% Date of presentation (\today or a fixed value): \date{date for everywhere} or \date[date for footline]{date for titlepage}
\date{\today} % 22. März 2022
% Impressum for thankyou slide (leave one empty if not wanted or needed):
\email{qstickle@hs-mittweida.de} % \email{E-Mail}
%\phone{+49 3727 58-1598} % \phone[Mobile Number]{Telephone Number}
%\office{Haus 8 | Richard Stücklen-Bau | Raum 8-207\newline Am Schwanenteich 6b | 09648 Mittweida} % \office{Office}
%\courseofstudies{Informatik (IF10w1-M)} % \courseofstudies{Course of Studies (Group Number)}
%\additional[Sidebar Text]{Main Text} % Additional information you may want to give (Department, Module, etc.) \additional[Sidebar Text]{Main Text}

%%%%%%%%%%%%%%%%%%%%%%%%%%%%%%%%%%%%%%%%%%
% 3) Use features
% Titlegraphic changes the title page to "wide" (default) if left empty or inserts given image (by file path) and scales to 6.2cm height otherwise:
\titlegraphic{} %\titlegraphic{figures/thankyou.jpg}

%%%%%%%%%%%%%%%%%%%%%%%%%%%%%%%%%%%%%%%%%%
% 4) Add bibliography
% Load the package and options you like, e.g. for recommended ieee style in combination of biblatex and biber:
%\usepackage[backend=biber, bibstyle=ieee, citestyle=numeric-comp, sorting=none, natbib=true, hyperref=true, dashed=false]{biblatex}
% Add your bibliography file(s)
%\addbibresource{literature.bib} 
% Dont forget to use \makebibliography where you want to put it, e.g. at the end of your presentation, after the "normal" slides:
% \appendix
% \makebibliography
% Compile with following command sequence to fully include bibliography: pdflatex, biber, pdflatex, pdflatex

%%%%%%%%%%%%%%%%%%%%%%%%%%%%%%%%%%%%%%%%%%
% 5) Existing macro usage examples

% \appendix is used as end marker for slides (slide numbers and progressbar)
% \makethankyou creates a thank you slide and is used as end marker for slides (progressbar)
% \makebibliography creates one or multiple bibliography slides

% For multiple speakers you can use \setcurrentspeaker{speaker name} or \setcurrentspeaker*{speaker name} prior to the frame
% To reset to the default author, use \resetcurrentspeaker{} or \resetcurrentspeaker*{}
% Starred versions of these macros prepend the word "speaker"

%%%%%%%%%%%%%%%%%%%%%%%%%%%%%%%%%%%%%%%%%%
% 6) Import additional packages you need, (re-)define macros and create a wonderful latex presentation

% You can remove this package - it is only needed for the dummy content
\usepackage{blindtext}

%%%%%%%%%%%%%%%%%%%%%%%%%%%%%%%%%%%%%%%%%%
\begin{document}

	\section{Grundlegende Verwendung}

	\begin{frame}[fragile]{General Information}{What is it?}
		Blablabla

		\scriptsize
		\begin{verbatim}
		\documentclass[aspectratio=169,onlytextwidth,t]{beamer}
		\usetheme{hsmw}

		\title[Kurztitel für Fußzeile]{Titel der Präsentation für Titelseite}
		\subtitle{Untertitel für Titelseite}
		\author{Name Vortragende(r)}
		\date{Datum der Präsentation}

		\begin{document}
			\begin{frame}{Erste Folie}
				Inhalt
			\end{frame}	
		\end{document}
		\end{verbatim}
	\end{frame}

	\begin{frame}[fragile]{Star Wars (1/3)}{Grundlegende Stilelemente beeinflussen}
		Sie können grundlegende Stilelemente über die Angabe von Optionen beim Laden des \textit{Beamer-Themes} (\texttt{\textbackslash{}usetheme[\textcolor{hsmw}{optionen}]\{hsmw\}}) beeinflussen:
		\begin{itemize}
			\item Allgemeine Anpassungen (Farben und Texte):
			\begin{itemize}
				\item Farbschema:
				\begin{itemize}
					\item Standardeinstellung Hochschul-Blau: \textcolor{hsmw}{hs}, \textcolor{hsmw}{inst} oder \textit{<leer>}
					\item Fakultäts-Farben: \textcolor{hsmw}{inw}, \textcolor{hsmw}{cb}, \textcolor{hsmw}{me}, \textcolor{hsmw}{sw}, \textcolor{hsmw}{wi}
					\item Alternative Schreibweise: \textcolor{hsmw}{faculty=inw}, \textcolor{hsmw}{faculty=cb}, \dots
				\end{itemize}

				\item Sprache (Ändert einige Bezeichner, z.\,B. Fakultätsnamen, Datumsformat, \dots): 
				\begin{itemize}
					\item Standardeinstellung Deutsch: \textcolor{hsmw}{language=ngerman} oder \textit{<leer>}
					\item Englisch: \textcolor{hsmw}{language=english}
					\item Andere Sprachen sind möglich, aber noch nicht implementiert (bei Interesse melden!)
				\end{itemize}
			\end{itemize}

			\item Individuelle Anpassungen an den eigenen Geschmack:
			\begin{itemize}
				\item \textcolor{hsmw}{toc} fügt ein Inhaltsverzeichnis nach der Titelseite ein
				\item \textcolor{hsmw}{sectionslide} fügt eine zusätzliche Seite für jede \verb!\section! ein
			\end{itemize}
		\end{itemize}
	\end{frame}

	\begin{frame}[fragile]{Mögliche Optionen (2/3)}{Grundlegende Stilelemente beeinflussen}
		\begin{itemize}
			\item[] \textcolor{hsmwLightGray}{Individuelle Anpassungen an den eigenen Geschmack (Fortsetzung):}
			\begin{itemize}
				\item \textcolor{hsmw}{subsectionslide} fügt eine zusätzliche Seite für jede \verb!\subsection! ein.\newline
					(kombinierbar mit \textcolor{hsmw}{sectionslide})
				\item \textcolor{hsmw}{smallpagenumber} reduziert die Größe der Foliennummer
				\item \textcolor{hsmw}{nototalpages} verbirgt die Gesamtzahl der Folien neben der Foliennummer
				\item \textcolor{hsmw}{nofacultyicon} verbirgt das Fakultäts-Icon auf der Titelseite
				\item \textcolor{hsmw}{colormath} hebt mathematische Formeln farbig ab
				\begin{itemize}
					\item Keine Hervorhebung: \textcolor{hsmw}{colormath=off} oder \textit{<leer>}
					\item Formeln sind hervorgehoben, aber Text in Formeln mittels \verb!\text{...}! wird wie normaler Text dargestellt: \textcolor{hsmw}{colormath=nottext} oder \textcolor{hsmw}{colormath}
					z.\,B.
					\[\sum^n_{k=0} k = \frac{n(n+1)}2 \qquad \text{ für }  n \in\mathbb N\]
					\item Vollständige Hervorhebung: \textcolor{hsmw}{colormath=full}
				\end{itemize}
			\end{itemize}
		\end{itemize}
	\end{frame}
	
	\begin{frame}[fragile]{Mögliche Optionen (3/3)}{Grundlegende Stilelemente beeinflussen}
		\begin{itemize}
			\item[] \textcolor{hsmwLightGray}{Individuelle Anpassungen an den eigenen Geschmack (Fortsetzung):}
			\begin{itemize}
				\item \textcolor{hsmw}{printhandout} setzt zwei Folien pro A4-Seite (Klassenoption \textit{handout} notwendig)
				\item \textcolor{hsmw}{noframesubtitle} deaktiviert alle Untertitel der Folien (vergrößert Textbereich)
			\end{itemize}
			\item Zusätzliche Optionen (nicht ganz konform mit dem Corporate Design):
			\begin{itemize}
				\item \textcolor{hsmw}{titlepagedate} zeigt das Datum auf der Titelseite
				\item \textcolor{hsmw}{fancystyle} aktiviert einige Farbanpassungen
				\item \textcolor{hsmw}{progressbar} zeigt einen Fortschrittsbalken am unteren Rand der Folien
			\end{itemize}
		\end{itemize}

		\vspace*{\baselineskip}
		\textbf{Beispiele:}
		\begin{itemize}
			\item \verb!\usetheme[toc, sectionslide, nofacultyicon]{hsmw}!
			\item \verb!\usetheme[cb, colormath, progressbar]{hsmw}!
			\item \verb!\usetheme[faculty=wi, colormath]{hsmw}!
		\end{itemize}
	\end{frame}

	\begin{frame}[fragile]{Zusätzliche Befehle}{Folien erstellen oder Werte manipulieren}
		\begin{itemize}
			\item Grafik auf der Titelseite einfügen: \verb!\titlegraphic{pfad/zum/bild.jpg}!

			\item Bibliographie hinzufügen
			\begin{itemize}
				\item Literaturdateien in Präambel einfügen: \verb!\addbibresource{literature.bib}!
				\item Literaturverzeichnis im Textkörper einbinden: \verb!\makebibliography!
			\end{itemize}

			\item Abschlussfolie hinzufügen
			\begin{itemize}
				\item Kontaktdaten in Präambel ergänzen: \verb!\email{E-Mail}! ...
				\item Abschlussfolie im Textkörper einbinden: \verb!\makethankyou!
			\end{itemize}

			\item Aktuellen Sprecher auf Folie anzeigen (bei mehreren Präsentierenden)
			\begin{itemize}
				\item Vor der Folie den Sprecher setzen: \verb!\setcurrentspeaker{Name des Sprechers}!
				\item Nach der Folie wieder zurücksetzen: \verb!\resetcurrentspeaker{}!
				\item Mittels Stern-Befehlen das Wort \enquote{Sprecher} voranstellen: \verb!\setcurrentspeaker*{Name des Sprechers}!
			\end{itemize}
		\end{itemize}
	\end{frame}
	
	\begin{frame}[fragile]{Erweitertes Beispiel (Stil: Fakultät CB)}{Mit ein paar zusätzlichen Optionen und Befehlen}
		\scriptsize
		\begin{verbatim}
			\documentclass[aspectratio=169,onlytextwidth,t]{beamer}
			\usetheme[cb, colormath]{hsmw}
			
			\title[Kurztitel für Fußzeile]{Titel der Präsentation für Titelseite}
			\subtitle{Untertitel für Titelseite}
			\author{Name Vortragende(r)}
			\email{E-Mail}
			\titlegraphic{figures/thankyou.jpg}

			\begin{document}
				\begin{frame}{Erste Folie}{Mit Untertitel}
					Inhalt
				\end{frame}	

				\appendix
				\makethankyou
			\end{document}
		\end{verbatim}
	\end{frame}

	\begin{frame}{Anwendungshinweise}{Was es zu beachten gilt}
		\begin{itemize}
			\item Für die Verwendung in lokalen TeX-Distributionen oder auch Overleaf geeignet

			\item Verzeichnisstruktur für das Auffinden der Dateien notwendig
			\begin{itemize}
				\item Funktionen und Aufbau auf mehrere Quelldateien verteilt
				\begin{itemize}
					\item beamerthemehsmw.sty: Optionen, Pakete und Macros (lädt die restlichen Dateien)
					\item beamerouterthemehsmw.sty: Allgemeine Layout-Einstellungen (Folientitel, Fußzeilen, ...)
					\item beamerinnerthemehsmw.sty: Inhaltsbezogene Layout-Einstellungen (Titelseite, Aufzählungen, ...)
					\item beamerfontthemehsmw.sty: Die verwendeten Schriftstile und -größen
					\item beamercolorthemehsmw*.sty: Das spezifische Farbschema für die einzelnen Elemente (inkl. Fakultätsfarben)
				\end{itemize}

				\item Unterverzeichnis für zusätzliches Bildmaterial: ./figures/*
			\end{itemize}
		\end{itemize}
	\end{frame}

	\begin{frame}{Besonderheiten}{Eventuelle Probleme, die gar keine sind}
		\begin{itemize}
			\item Bei überlangen (Unter-)Titeln auf der Titelseite und auf den Folien wird bei Bedarf die Schriftgröße heruntergesetzt
			\item Sie erhalten dafür eine Paket-Warnung in der Logdatei, die Sie darauf hinweist:
			\newline
			\resizebox{\linewidth}{!}{\enquote{Package beamerinnerthemehsmw Warning: Font of text '\textit{<text>}' is scaled down by a factor of \textit{<factor>}}}
			\item Sie können diese Texte ggf. anpassen, damit sie nicht skaliert werden müssen
			\item Sie können den Warnhinweis allerdings auch einfach ignorieren
		\end{itemize}
	\end{frame}

	\section{Inhalte gestalten}

	\begin{frame}{Eine normale Folie mit Fließtext}{... und einem Untertitel}
		\blindmathtrue
		\blindtext
	\end{frame}

	\begin{frame}[c]{Eine normale Folie vertikal zentriert}{Unter Verwendung der Folien-Option: \texttt{\textbackslash{}begin\{frame\}[c] ... \textbackslash{}end\{frame\}}}
		\blindmathtrue
		\blindtext
	\end{frame}

	\begin{frame}[b]{Eine normale Folie unten ausgerichtet}{Unter Verwendung der Folien-Option: \texttt{\textbackslash{}begin\{frame\}[b] ... \textbackslash{}end\{frame\}}}
		\blindmathtrue
		\blindtext
	\end{frame}
	
	\begin{frame}{Eine Folie mit zwei Spalten}{Einfache Mathematik: Mehr Spalten = mehr Platz}
		\begin{columns}
			\begin{column}[T]{.5\textwidth}
				\blindlistlist[3]{itemize}[3]
			\end{column}
			\begin{column}[T]{.5\textwidth}
				\blindlistlist[2]{itemize}[4]
			\end{column}
		\end{columns}
	\end{frame}

	\begin{frame}{Eine Folie mit zwei Spalten}{Auch passend für Abbildungen}
		\begin{columns}
			\begin{column}[T]{.5\textwidth}
				\blindlistlist[3]{itemize}[3]
			\end{column}
			\begin{column}[T]{.5\textwidth}
				\centering
				\includegraphics[width=\textwidth]{figures/thankyou.jpg}
				\captionof{figure}{Das Bild der Danke-Seite}
				\label{fig:thankyou}
			\end{column}
		\end{columns}
	\end{frame}

	\begin{frame}[fragile]{Inhalte absolut positionieren}
		\begin{itemize}
			\item Der Koordinatenursprung für ist die obere linke Ecke
			\item Koordinatensystem in Zentimetern, an Seitenverhältnis 16:9 ausgerichtet
			\begin{itemize}
				\item $ 0 \le x \le 16 $
				\item $ 0 \le y \le 9 $
			\end{itemize}

			\item Verwendung von \verb!begin{textblock}{breite}(x-pos, y-pos)!...
		\end{itemize}

		\begin{textblock}{5}(10, 4.5)
			\scriptsize
			\begin{verbatim}
				\begin{textblock}{5}(10, 4.5)
					Inhalt
				\end{textblock}
			\end{verbatim}
		\end{textblock}

		\begin{textblock}{15}(0.5, 6)
			\hrule
			\scriptsize
			\begin{verbatim}
				\begin{textblock}{15}(0.5, 6)
					\hrule
				\end{textblock}
			\end{verbatim}
		\end{textblock}
	\end{frame}

	\begin{frame} 
		\frametitle{Mehrere Folien mittels \textit{Overlays}} 
		\begin{theorem}
			Es gibt keine \enquote{größte} Primzahl.
		\end{theorem} 
		\begin{enumerate} 
			\item<1-| alert@1> Angenommen $p$ wäre die größte Primzahl.
			\item<2-> Sei $q$ das Produkt der ersten $p$ Zahlen. 
			\item<3-> Dann ist $q+1$ durch keine davon teilbar.
			\item<4-> Aber $q + 1$ ist größer als $1$ und daher durch eine Primzahl teilbar, die nicht in den ersten $p$ Zahlen liegt.
		\end{enumerate}

		\uncover<3->{\scriptsize Hinweis: Mathe ist super kompliziert!}

		\vfill
		\only<2,4>{\centering\textbf{Achtung}: Besonders wichtiger Schritt!}
		\vfill
	\end{frame}

	\section{Benutzerdefinierte Anpassungen}

	\newcommand{\rgb}[1]{\textcolor{hsmw80}{#1}}
	\newcommand{\cmd}[1]{\rgb{\textbackslash{}#1}}
	\newcommand{\textto}{\hspace*{0.2ex}\tikz[baseline=-.33em] \draw[-latex] (0,0) to ++(2ex, 0);\hspace*{0.2ex}}
	\begin{frame}{Zusätzliche, beeinflussbare Macros}{... und deren Abhängigkeiten (zur Feinabstimmung der eigenen Präsentation)}
		\begin{itemize}
			\item Option \rgb{hs}, \rgb{cb}, ... oder \rgb{faculty=cb, me, ...} (Farbschema)
			\begin{itemize}
				\item \cmd{insertfacultyicon} (Titelseite)
				\item \cmd{insertfacultyname} (Option \rgb{language}) \textto{} \cmd{institute\{\cmd{insertfacultyname}\}}
			\end{itemize}
			\item \cmd{insertthankyoutitle}, \cmd{insertthankyoutext}, \cmd{insertthankyousidebartext}
			\begin{itemize}
				\item \cmd{email} \textto{} \cmd{insertemail}
				\item \cmd{phone} \textto{} \cmd{insertmobilephone}, \cmd{inserttelephone}
				\item \cmd{office} \textto{} \cmd{insertoffice}
				\item \cmd{courseofstudies} \textto{} \cmd{insertcourseofstudies}
				\item \cmd{additional} \textto{} \cmd{insertadditionalsidebar}, \cmd{insertadditional}
			\end{itemize}
			\item \cmd{setcurrentspeaker}, \cmd{resetcurrentspeaker} (\cmd{insertshortauthor})
			\begin{itemize}
				\item Stern-Version (\cmd{setcurrentspeaker*}) setzt zusätzlich Label (Option \rgb{language})
				\item \cmd{currentspeaker} (\cmd{currentspeakerlabel}) \textto{} \cmd{insertcurrentspeaker}
			\end{itemize}
		\end{itemize}
	\end{frame}

	\section{FAQ}

	\begin{frame}[c]{FAQ: Häufig gestellte Fragen}{Hier ist noch Platz für Anwendungsfälle oder Antworten auf häufig gestellte Fragen}
		Es sind alle zum Testen und zur Übermittlung von konstruktivem Feedback eingeladen!
		\\[\baselineskip]
		Bei Ideen, Wünschen, Anregungen, Fragen und auch Problemen:
		\begin{itemize}
			\item Offizielle LaTeX-GitLab-Gruppe der Hochschule Mittweida:
			\newline
			\href{https://git.hs-mittweida.de/hsmw-latex}{git.hs-mittweida.de/hsmw-latex}

			\item Kontaktieren Sie mich gern per E-Mail \href{mailto:schildba@hs-mittweida.de?subject=[LaTeX] Beamer-Vorlage}{schildba@hs-mittweida.de}

			\item Nutzen Sie einen der anderen verfügbaren Kommunikationskanäle
		\end{itemize}
	\end{frame}

	\appendix
	\makethankyou
%	\makebibliography

	\section{\appendixname}

	\begin{frame}{Zusätzliche Folien}{Der Anhang zählt nicht mit zu den regulären Folien}
		\blindtext
	\end{frame}

\end{document}
