\documentclass[
	aspectratio=169,	% Modern aspect ratio (TODO: Other ratios not yet supported)
	onlytextwidth,		% Sets totalwidth=\textwidth and therefore e.g. columns won't invade the margins
	t,					% Default vertical alignment of frames and colums at top (default is centered) % Stored in \beamer@centered (\beamer@centeredfalse, \beamer@centeredtrue)
%	handout,			% Create a basic handout of the presentation (removes overlays)
	]{beamer}

%%%%%%%%%%%%%%%%%%%%%%%%%%%%%%%%%%%%%%%%%%
% 1) Load the desired presentation theme
\usetheme[
% Individual options to customize the presentation conforming to the corporate design
	cb,					% Change default faculty color set and predefined faculty values: hs or <empty> (default), inw, cb, me, sw, wi, inst (\renewcommand{\insertfacultyname}{Institutname} needed)
	language=english,	% Change language to english (default: ngerman), other languages are possible (see babel-package) but may need further adjustments 
	toc,				% Adds a ToC slide
%	sectionslide,		% Display separate section slides
%	subsectionslide,		% Display separate subsection slides containing section and subsection name
%	smallpagenumber,	% Reduces the size of the page number
%	nototalpages,		% Hides the total number of slide in footline
%	nofacultyicon,		% Hides the faculty icon on title page
	colormath=nottext,	% Enables coloring of math text: off or <empty>, full, nottext (default)
%	printhandout,		% Places two slides on a single a4 paper for printing the presentation (beamer class option 'handout' needed)
%	noframesubtitle,	% Disables frame subtitles alltogether and slightly increases frame height
% Additional style options not completely conforming to the corporate design
	titlepagedate,		% Shows the date on the titelpage
%	fancystyle,			% Enables some fancy styles, that are not part of the corporate design specifications (default: off)
	progressbar,		% Shows the progressbar in footline (run twice to update progressbar)
	]{hsmw} 

%%%%%%%%%%%%%%%%%%%%%%%%%%%%%%%%%%%%%%%%%%
% 2) Specify default fields for presentation and pdf document properties
% Set the title: \title{Long title everywhere} or \title[Short title for footline]{Long title for titlepage}
\title[RC4-Algorithm]{RC4-Algorithm}
% Authors (separate multiple author names e.g. with \and for additional space): \author{author for everywhere} or \author[author for footline]{author for titlepage and thankyouslide}
\author[Quentin Stickler]{Quentin Stickler, B.Sc.}
% Institute (will be prefilled automatically, depending on chosen faculty theme option): \institute{institute for everywhere} or \institute[institute for footline and thankyouslide]{institute for titlepage}
%\institute[institute for footline and thankyouslide]{institute for titlepage}
% Date of presentation (\today or a fixed value): \date{date for everywhere} or \date[date for footline]{date for titlepage}
\date{24. April 2024} % 22. März 2022
% Impressum for thankyou slide (leave one empty if not wanted or needed):
\email{qstickle@hs-mittweida.de} % \email{E-Mail}
%\phone{+49 3727 58-1598} % \phone[Mobile Number]{Telephone Number}
%\office{Haus 8 | Richard Stücklen-Bau | Raum 8-207\newline Am Schwanenteich 6b | 09648 Mittweida} % \office{Office}
%\courseofstudies{Informatik (IF10w1-M)} % \courseofstudies{Course of Studies (Group Number)}
%\additional[Sidebar Text]{Main Text} % Additional information you may want to give (Department, Module, etc.) \additional[Sidebar Text]{Main Text}

%%%%%%%%%%%%%%%%%%%%%%%%%%%%%%%%%%%%%%%%%%
% 3) Use features
% Titlegraphic changes the title page to "wide" (default) if left empty or inserts given image (by file path) and scales to 6.2cm height otherwise:
\titlegraphic{} %\titlegraphic{figures/thankyou.jpg}

%%%%%%%%%%%%%%%%%%%%%%%%%%%%%%%%%%%%%%%%%%
% 4) Add bibliography
% Load the package and options you like, e.g. for recommended ieee style in combination of biblatex and biber:
%\usepackage[backend=biber, bibstyle=ieee, citestyle=numeric-comp, sorting=none, natbib=true, hyperref=true, dashed=false]{biblatex}
% Add your bibliography file(s)
%\addbibresource{literature.bib} 
% Dont forget to use \makebibliography where you want to put it, e.g. at the end of your presentation, after the "normal" slides:
% \appendix
% \makebibliography
% Compile with following command sequence to fully include bibliography: pdflatex, biber, pdflatex, pdflatex

%%%%%%%%%%%%%%%%%%%%%%%%%%%%%%%%%%%%%%%%%%
% 5) Existing macro usage examples

% \appendix is used as end marker for slides (slide numbers and progressbar)
% \makethankyou creates a thank you slide and is used as end marker for slides (progressbar)
% \makebibliography creates one or multiple bibliography slides

% For multiple speakers you can use \setcurrentspeaker{speaker name} or \setcurrentspeaker*{speaker name} prior to the frame
% To reset to the default author, use \resetcurrentspeaker{} or \resetcurrentspeaker*{}
% Starred versions of these macros prepend the word "speaker"

%%%%%%%%%%%%%%%%%%%%%%%%%%%%%%%%%%%%%%%%%%
% 6) Import additional packages you need, (re-)define macros and create a wonderful latex presentation

% You can remove this package - it is only needed for the dummy content
\usepackage{blindtext}
\usepackage{pythonhighlight}
\usepackage{amsmath}
\usepackage{setspace}
\usepackage{listings}
\usepackage{xcolor}
\usepackage[natbib=true, backend=biber, style=authoryear, useprefix=true]{biblatex}
\addbibresource{sources.bib}
\AtBeginBibliography{\small}

\definecolor{codegreen}{rgb}{0,0.6,0}
\definecolor{codegray}{rgb}{0.5,0.5,0.5}
\definecolor{codepurple}{rgb}{0.58,0,0.82}
\definecolor{backcolor}{rgb}{0.95,0.95,0.92}

\lstdefinestyle{Pythonstyle}{
	language=Python,
	backgroundcolor=\color{backcolor},
	commentstyle=\color{codegreen},
	keywordstyle=\color{magenta},
	numberstyle=\tiny\color{codegray},
	stringstyle=\color{codepurple},
	basicstyle=\ttfamily\footnotesize,
	breakatwhitespace=false,
	breaklines=true,
	captionpos=b,
	keepspaces=true,
	numbers=left,
	numbersep=5pt,
	showspaces=false,
	showstringspaces=false,
	showtabs=false,
	showtabs=false,
	tabsize=2
}
\lstset{style=Pythonstyle}

\newtheorem{thm}{Theorem}

%%%%%%%%%%%%%%%%%%%%%%%%%%%%%%%%%%%%%%%%%%
\begin{document}
\section{General info}

\begin{frame}[fragile]{General info}{History}
	\begin{itemize}
		\item Stream cipher with \textbf{variable} key-size length
		\item Used to be the most widely used stream cipher in software applications
		\item Invented in 1987 by Ron Rivest
		\item Kept secret but got leaked in 1994
		\item \textbf{Easy} to implement and quite \textbf{fast} (Encryption up to 10x faster than DES)
		\item Offers a \textbf{lot of weaknesses und vulnerabilities}
		\item Better alternatives have been invented
		\item Now only used in private projects due to its simplicity and performance
	\end{itemize}
\end{frame}

\section{RC4 Algorithm in detail}

\begin{frame}[fragile]{RC4 Algorithm}{How does it work?}
	\begin{itemize}
		\item Consists of two parts
		\item Part 1: Key Scheduling Algorithm \textbf{(KSA)}
		\item Part 2: Pseudo Random Number Generator Algorithm \textbf{(PRGA)}
		\item Used in the algorithm:
		\item S-Box (Array) with length of $256$
		\item K-Box with repeating key entries
		\item Two 8-byte sized counters $i$ and $j$
	\end{itemize}
\end{frame}

\begin{frame}[fragile]{Initialization}{Part One: Filling S-Box and T-Box}
	\begin{itemize}
		\item Counters $i$ and $j$ set to $0$
		\item Linear filling of the S-Box from $0$ to $255$ ($S[0] = 0$, $S[1] = 1$\dots)
		\item Following loop will be run:
		\item State space thus: $(2^{8})^2 * 256! \approx 2^{1700}$ (Question 6)
	\end{itemize}
	\begin{lstlisting}[language=Python]
		for x in range(256):
			sbox[x] = x
			kbox[x] = key[x % len(key)]
	\end{lstlisting}
\end{frame}


\begin{frame}{Initialization}{Example}
	\begin{itemize}
		\item Text = "TestText"
		\item Key = "TestKey"
		\item S-Box = $[0, 1, 2, 3 \dots, 255]$
		\item Initialization of T-Box:
		\begin{itemize}
			\item Keylength = 7
			\item Ascii-Text = $84 101 115 116 75 101 121$
		\end{itemize}
		\medskip
		$\begin{array}{|ccccccc|}
		84 & 101 & 115 & 116 & 75 & 101 & 121 \\
		84 & 101 & 115 & 116 & 75 & 101 & 121 \\
		\dots & \dots & \dots & \dots & \dots & \dots & \dots \\
		\dots & \dots & \dots & 84 & 101 & 115 & 116 \\
		\end{array}$
	\end{itemize}
\end{frame}


\begin{frame}[fragile]{Initialization}{Part Two: Permutation}
	\begin{itemize}
		\item Permutate S-Box based on given key
		\item We always use modulo $n = 256$ because of the given length
		\begin{lstlisting}[language=Python]
			j = 0
			for i in range(256):
				j = (j + sbox[i] + kbox[i]) % 256
				Swap(sbox[i], sbox[j])
			return sbox
		\end{lstlisting}
		\item At the end: (Pseudo-)randomly generated S-Box
		
	\end{itemize}
\end{frame}


\begin{frame}{Permutation Example}{i = 0}
	\begin{itemize}
		\item S-Box Initialization:
		\medskip
		$\begin{array}{|ccccccc|}
		\textcolor{green}{0} & 1 & 2 & 3 & 4 & 5 & 6 \\
		\dots & \dots & \dots & \dots & \dots & \dots & \dots \\
		249 & 250 & 251 & 252 & 253 & 254 & 255 \\
		\end{array}$
		\item i = \textcolor{green}{0}
		\item j = (\textcolor{blue} {j} + \textcolor{green}{S[i]} + \textcolor{red}{T[i]}) $\%$ (256)
		\item j = (\textcolor{blue} {84} + \textcolor{green} {0} + \textcolor{red} {84}) $\%$(256) = 168 $\%$ (256) = $168$
		\item Swap S[i] (0) and S[j] (84)
		\item S[i] = $84$, S[j] = $0$
	\end{itemize}
\end{frame}

\begin{frame}{Permutation Example Continued}{i = 1}
	$\begin{array}{|ccccccc|}
		84 & \textcolor{green}{1} & 2 & 3 & 4 & 5 & 6 \\
		\dots & \dots & \dots & \dots & \dots & \dots & \dots \\
		80 & 81 & 82 & 83 & 0 & 85 & 86 \\
		\dots & \dots & \dots & \dots & \dots & \dots & \dots \\
		249 & 250 & 251 & 252 & 253 & 254 & 255 \\
	\end{array}$
	\begin{itemize}
		\item i = \textcolor{green}{1}
		\item j = (\textcolor{blue} {j} + \textcolor{green}{S[i]} + \textcolor{red}{T[i]}) $\%$ (256)
		\item j = (\textcolor{blue} {186} + \textcolor{green} {1} + \textcolor{red} {101}) $\%$ (256) = 288 $\%$ (256) = $32$
		\item Swap S[i] (1) and S[j] (186)
		\item S[i] = $186$, S[j] = $1$
	\end{itemize}
\end{frame}

\begin{frame}{Permutation Example Continued}{i = 2}
	$\begin{array}{|ccccccc|}
		84 & 186 & \textcolor{green}{2} & 3 & 4 & 5 & 6 \\
		\dots & \dots & \dots & \dots & \dots & \dots & \dots \\
		80 & 81 & 82 & 83 & 0 & 85 & 86 \\
		\dots & \dots & \dots & \dots & \dots & \dots & \dots \\
		249 & 250 & 251 & 252 & 253 & 254 & 255 \\
	\end{array}$
	\begin{itemize}
		\item i = \textcolor{green}{2}
		\item j = (\textcolor{blue} {j} + \textcolor{green}{S[i]}) + \textcolor{red}{T[i]} $\%$ (256)
		\item j = (\textcolor{blue} {47} + \textcolor{green} {2} + \textcolor{red} {115}) $\%$ (256) = 126 $\%$ (256) = $126$
		\item Swap S[i] (1) and S[j] (47)
		\item S[i] = $47$, S[j] = $2$
	\end{itemize}
\end{frame}

\begin{frame}{Permutation Example}{Final S-Box Form}
	\tiny
	\smallskip
	\centering
	$\begin{array}{|cccccccccccccccc|}
		84& 186& 47& 208& 12& 95& 222& 212& 71& 9& 26& 246& 103& 38& 28& 165 \\
		138 & 68 & 130 & 10 & 50 & 143 & 72 & 155 & 39 & 139 & 112 & 16 & 79 & 78 & 196 & 146 \\
		216 & 179 & 159 & 178 & 34 & 119 & 59 & 56 & 63 & 183 & 53 & 197 & 100 & 236 & 101 & 4 \\
		176 & 250 & 116 & 67 & 5 & 60 & 194 & 35 & 105 & 87 & 118 & 218 & 97 & 168 & 1 & 77 \\
		44 & 229 & 25 & 48 & 141 & 42 & 175 & 91 & 94 & 211 & 121 & 169 & 215 & 89 & 99 & 24 \\
		98 & 164 & 181 & 129 & 255 & 185 & 110 & 8 & 220 & 154 & 109 & 219 & 201 & 153 & 120 & 62 \\
		51 & 0 & 217 & 37 & 20 & 226 & 43 & 127 & 170 & 227 & 243 & 249 & 133 & 126 & 161 & 156 \\
		82 & 167 & 140 & 115 & 145 & 74 & 182 & 83 & 184 & 104 & 189 & 81 & 52 & 233 & 172 & 245 \\
		157 & 66 & 124 & 177 & 102 & 80 & 147 & 171 & 106 & 162 & 70 & 30 & 199 & 6 & 69 & 18 \\
		173 & 45 & 32 & 88 & 125 & 221 & 7 & 65 & 75 & 158 & 232 & 128 & 237 & 190 & 108 & 248 \\
		13 & 144 & 2 & 46 & 49 & 31 & 134 & 123 & 92 & 40 & 114 & 254 & 131 & 213 & 41 & 93 \\
		117 & 253 & 23 & 137 & 234 & 209 & 224 & 136 & 107 & 90 & 202 & 223 & 132 & 27 & 15 & 207 \\
		73 & 195 & 239 & 64 & 206 & 251 & 149 & 228 & 231 & 166 & 187 & 214 & 86 & 242 & 191 & 76 \\
		192 & 58 & 142 & 61 & 57 & 193 & 33 & 244 & 180 & 205 & 111 & 3 & 122 & 36 & 22 & 14 \\
		240 & 252 & 238 & 188 & 247 & 85 & 203 & 174 & 200 & 11 & 148 & 152 & 160 & 230 & 210 & 29 \\
		96 & 235 & 163 & 150 & 17 & 204 & 54 & 55 & 198 & 151 & 225 & 21 & 135 & 113 & 19 & 241 \\
	\end{array}$
	\medskip
	\normalsize
	\begin{itemize}
		\item Result = Permutated S-Box
		\item All numbers from $0-255$ in "random" places
	\end{itemize}
\end{frame}

\begin{frame}[fragile]{Keystream Generator}{Python Code}
	\begin{itemize}
		\item Generate keystream depending on length of given plaintext
		\begin{lstlisting}[language=Python]
			keystream = []
			i = 0
			j = 0
			for x in range(len(text)):
				i = (1 + i) % 256
				j = (sbox[i] + j) % 256
				
				Swap(sbox[i], sbox[j])
				keystream.append(sbox[(sbox[i] + sbox[j]) % 256])    
			return keystream   
		\end{lstlisting}
	\end{itemize}
\end{frame}

\begin{frame}[fragile]{Keystream Generator}{Example, i = 0}
	\begin{itemize}
		\item i = $0$, j = $0$
		\item i = (0 + 1) $\%$ 256 = $1$
		\item j = (0 + 186) $\%$ 256 = 186 $\%$ 256 = $186$
		\item Swap S[i] ($186$) and S[j] ($202$)
		\item t = (202 + 186) $\%$ 256 = 388 $\%$ 256 = $132$
		\item S[t] = $102$
		\item keystream = [$102$, ]
	\end{itemize}
\end{frame}

\begin{frame}[fragile]{Keystream Generator}{Example, i = 1}
	\begin{itemize}
		\item i = $1$, j = $186$
		\item i = (1 + 1) $\%$ 256 = $2$
		\item j = (186 + 47) $\%$ 256 = 233 $\%$ 256 = $233$
		\item Swap S[i] ($47$) and S[j] ($11$)
		\item t = (47 + 11) $\%$ 256 = 58 $\%$ 256 = $58$
		\item S[t] = $118$
		\item keystream = [$102$, $118$, ]
	\end{itemize}
\end{frame}

\begin{frame}[fragile]{Keystream Generator}{Example, i = 2}
	\begin{itemize}
		\item i = $2$, j = $233$
		\item i = (2 + 1) $\%$ 256 = $3$
		\item j = (233 + 208) $\%$ 256 = 451 $\%$ 256 = $185$
		\item Swap S[i] ($208$) and S[j] ($90$)
		\item t = (208 + 90) $\%$ 256 = 298 $\%$ 256 = $42$
		\item S[t] = $53$
		\item keystream = [$102$, $118$, $53$, ....]
		\item Final keystream = [$102$, $118$, $53$, $212$, $66$, $47$, $204$, $221$]
	\end{itemize}
\end{frame}

\begin{frame}[fragile]{Encryption}{Bytewise XOR}
	\small
	\begin{itemize}
		\item Plaintext = "TestText" = {\color{purple}[84, 101, 115, 116, 84, 101, 120, 116]}
		\item Keystream = {\color{olive}[102, 118, 53, 212, 66, 47, 204, 221]}
		\item Plaintext $\oplus$ Keystream =
		\item "0X320X130X460XA00X160X4A0XB40XA9" = [$50$, $19$, $70$,$160$, $22$, $74$, $180$, $169$]
	\end{itemize}
	\normalsize
\end{frame}

\begin{frame}[fragile]{Decryption}{Bytewise XOR}
	\small
	\begin{itemize}
		\item Ciphertext = "0X320X130X460XA00X160X4A0XB40XA9" = {\color{purple}[50, 19, 70, 160, 22, 74, 180, 169]}
		\item Keystream = {\color{olive}[102, 118, 53, 212, 66, 47, 204, 221]} 
		\item Ciphertext $\oplus$ Keystream = "Plaintext"
	\end{itemize}
	\normalsize
\end{frame}

\section{Attacking RC4}

\begin{frame}[fragile]{WEP}{Short summary}
	\begin{itemize}
		\item Wired Equivalent Protocol
		\item Used in IEEE 802.11 for protecting LAN users against casual eavesdropping
		\item Encrypt wirelessly transmitted packets
		\item Key used for encryption consists of a long-term key (root key) and an initialization vector
		\item $RC4Key = IV||rk$
		\item Different public IV per packet, 24-bit-sized; $IV=(X,Y,Z)$
		\item 40-bit-sized secret $rk$
	\end{itemize}
\end{frame}

\begin{frame}[fragile]{Security problems in WEP}{Outdated since 2004}
	\begin{itemize}
		\item "Swiss Cheese" of protocols $\rightarrow$ lots of security vulnerabilities
		\item Small key sizes; only 64-bit and 128-bit encryption key sizes
		\item CRC-32 for detecting changes made to data
		\begin{itemize}
		\item Useful for detecting errors but useless for validating cryptographic validation
		\item Attacker can easily alter the data so that the validation check is getting verified
		\end{itemize}
		\item Small IV sizes of 24-bit $\rightarrow 2^{24}$ possibilities ($<17 million$)
	\end{itemize}
\end{frame}

\begin{frame}[fragile]{Attacking RC4 in WEP}{Utilizing IVs}
	\begin{itemize}
		\item Small key sizes (40-bit $rk$ and 24-bit $IV$) \cite{stovsic2012rc4}
		\item IV is sent clearly together with packets
		\item Make use of "weak IVs" to recover first byte of every message
		\item \textbf{FMS attack}
	\end{itemize}
\end{frame}

\begin{frame}[fragile]{FMS attack on RC4}{General process}
	\begin{itemize}
		\item Cryptanalysis Trudy graps a lot of transfered data
		\item Tries to catch IVs of specific forms 
		\item Goal $\rightarrow$ Recover the long-term key $\rightarrow$ Then she can decrypt all the ciphertexts
		\item Example: $ IV = (3, N-1, V)$ , where $N-1 = 255, V$ any value ${1,\dots,255}$
		\item Long-term-key of the form $(3, 255, V, K\textsubscript{3}, K\textsubscript{4},K\textsubscript{5})$
		\item $K\textsubscript{3}, K\textsubscript{4},K\textsubscript{5}$ are the first unknown keybytes
		\item Clue is in the initialization phase
	\end{itemize}
\end{frame}

\begin{frame}[fragile]{Attacks on RC4}{Example for K\textsubscript{3}}
	\begin{itemize}
		\item Suppose, Trudy has recoverd $IV = (\textcolor{green}{3}, 255, V)$ 
		\item Used for recovering Example for $K\textsubscript{\textcolor{green}{3}}$
		\item Let's look at our S-Box during the initialization phase
		\item First, $S$ is set to the identitity permutation
		\medskip
		\begin{table}[h!]
			\begin{center}
			  \begin{tabular}{l|c|c|c|c|c|c|r}
				$i$ & 0 & 1 & 2 & 3 & 4 & 5 & \dots\\
				\hline
				$S\textsubscript{i}$ & 0 & 1 & 2 & 3 & 4 & 5 & \dots\\
			  \end{tabular}
			\end{center}
		  \end{table}

	\end{itemize}
\end{frame}

\begin{frame}[fragile]{Attacks on RC4}{Example for K\textsubscript{3}}
	\begin{itemize}
		\item Now, at the first step $i=0$, we compute the next $j$
		\item $j = j + S\textsubscript{i} + K\textsubscript{1} = 0 + 0 + 3$ $\% (256) = 3$
		\item Thus, the elements at position $S\textsubscript{i}$ and $S\textsubscript{j}$ are swapped
		\medskip
		\begin{table}[h!]
			\begin{center}
			  \begin{tabular}{l|c|c|c|c|c|c|c}
				$i$ & 0 & 1 & 2 & 3 & 4 & 5 & \dots\\
				\hline
				$S\textsubscript{i}$ & \textcolor{green}{3} & 1 & 2 & \textcolor{green}{0} & 4 & 5 & \dots\\
			  \end{tabular}
			\end{center}
		  \end{table}
		\item At the next step $i = 1$, we compute $j$ as
		\item $j = 3 + S\textsubscript{i} + K\textsubscript{i} = 3 + 1 + 255$ $\% (256) = 3$
	\end{itemize}
\end{frame}

\begin{frame}[fragile]{Attacks on RC4}{Example for K\textsubscript{3} Cont'd}

	\begin{table}[h!]
		\begin{center}
			\begin{tabular}{l|c|c|c|c|c|c|r}
			$i$ & 0 & 1 & 2 & 3 & 4 & 5 & \dots\\
			\hline
			$S\textsubscript{i}$ & 3 & \textcolor{green}{0} & 2 & \textcolor{green}{1} & 4 & 5 & \dots\\
			\end{tabular}
		\end{center}
	\end{table}

	\begin{itemize}
		\item At the next step $i = 2$, we compute $j$ as
		\item $j = 3 + S\textsubscript{2} + K\textsubscript{2} = 3 + 2 + V$ $\% (256) = 5+V$
	\end{itemize}

	\begin{table}[h!]
		\begin{center}
			\begin{tabular}{l|c|c|c|c|c|c|c|c|r}
			$i$ & 0 & 1 & 2 & 3 & 4 & 5 & \dots & 5 + V & \dots\\
			\hline
			$S\textsubscript{i}$ & 3 & 0 & \textcolor{green}{5 + V} & 1 & 4 & 5 & \dots & \textcolor{green}{2} & \dots\\
			\end{tabular}
		\end{center}
	\end{table}

\end{frame}

\begin{frame}[fragile]{Attacks on RC4}{Example for K\textsubscript{3}: Last step}

	
	\begin{itemize}
		\item At the next step $i = 3$, we compute $j$ as
		\item $j = 5 + V + S\textsubscript{3} + K\textsubscript{3} = 5 + V + 1 + K\textsubscript{3}$ $\%(256) = 6 + V + K\textsubscript{3}$
	\end{itemize}
	
	\begin{table}[h!]
		\begin{center}
			\begin{tabular}{l|c|c|c|c|c|c|c|c|c|c|r}
			$i$ & 0 & 1 & 2 & 3 & 4 & 5 & \dots & 5 + V & \dots & 6 + V + K\textsubscript{3} & \dots\\
			\hline
			$S\textsubscript{i}$ & 3 & 0 & 5 + V & \textcolor{green}{6 + V + K\textsubscript{3}} & 4 & 5 & \dots & 2 & \dots & \textcolor{green}{1} & \dots\\
			\end{tabular}
		\end{center}
	\end{table}

	\begin{itemize}
		\item Suppose$ S\textsubscript{0}, S\textsubscript{1}$ and $S\textsubscript{3}$ will remain unchanged until step $i = 255$
		\item Then, the first keystreambyte will be computed following the keystream generator algorithm
	\end{itemize}

\end{frame}

\begin{frame}[fragile]{Attacks on RC4}{Example for K\textsubscript{4} and K\textsubscript{5}: Last step}
	\begin{itemize}
		\item $IV = (4,255,V)$ for $K\textsubscript{4}$ after $i=4$ steps:
	\end{itemize}

	\small
	
	\begin{table}[h!]
		\begin{center}
			\begin{tabular}{l|c|c|c|c|c|c|c|c|c}
			$i$ & 0 & 1 & 2 & 3 & 4 & 5 & 6 + V & 9 + V + K3 & 10 + V + K3 + K4\\
			\hline
			$S\textsubscript{i}$ & 4 & 0 & 6 + V & 9 + V + K3 & \textcolor{green}{10 + V + K3 + K4} & 5 & 2 & 3 & 1\\
			\end{tabular}
		\end{center}
	\end{table}

	\normalsize

	\begin{itemize}
		\item $IV = (5,255,V)$ for $K\textsubscript{5}$ after $i=5$ steps:
	\end{itemize}

	\small

	\begin{table}[h!]
		\begin{center}
			\begin{tabular}{l|c|c|c|c|c}
			$i$ & 0 & 1 & 2 & 3 & 4\\
			\hline
			$S\textsubscript{i}$ & 5 & 0 & 7 + V & 10 + V + K3 + & 14 + V + K3 + K4\\
			\end{tabular}
		\end{center}
	\end{table}

	\begin{table}[h!]
		\begin{center}
			\begin{tabular}{c|c|c|c|c}
			5 & 7 + V & 10 + V + K3 & 14 + V + K3 + K4 & 15 + V + K3 + K4 + K5\\
			\hline
			\textcolor{magenta}{15 + V + K3 + K4 + K5} & 2 & 3 & 4 & 5\\
			\end{tabular}
		\end{center}
	\end{table}

	\normalsize

\end{frame}

\begin{frame}[fragile]{RC4 Attack}{Recover $K\textsubscript{3}$}

	\begin{lstlisting}[language=Python]
		keystream = []
		i = 0
		j = 0
		for x in range(len(text)):
			i = (i + 1) % 256
			j = (sbox[i] + j) % 256
			Swap(sbox[i], sbox[j])
			keystream.append(sbox[(sbox[i] + sbox[j]) % 256])    
		return keystream   
	\end{lstlisting}
	\begin{itemize}
		\item i = $1$, j = $0$
		\item $K\textsubscript{B} = (6 + V + K\textsubscript{3})$ $\% (256)$
	\end{itemize}

\end{frame}

\begin{frame}[fragile]{RC4 Attack}{Recover $K\textsubscript{3}$ Continued}

	\begin{itemize}
		\item $K\textsubscript{B} = (6 + V + K\textsubscript{3})$ $\% (256)$
		\item Suppose, Trudy can guess or knows the first byte of the plaintext, she can determine $K\textsubscript{3}$ with:
		\item $\rightarrow$ $K\textsubscript{3}  = K\textsubscript{B} - 6 - V$ $\% (256)$
	\end{itemize}
\end{frame}

\begin{frame}[fragile]{Recovery of unknwon bytes}{General approach} 
	\begin{thm}
		Let $ K\textsubscript{n} $ be the RC4 key value at position $ n $. Let $ IV\textsubscript{n} $ be a tuple of $ (n, N-1, V) $,
		where $ N = 256, V \in {{0,\dots,255}}, n \geq 3 $ and $ k\textsubscript{n} $ the known keystreambyte at position $n$. Then (Question 8): \\
		\begin{center}
			$K\textsubscript{n} = k\textsubscript{n} - \sum_{1}^{n}x - V - (\sum_{3}^{n-1}K\textsubscript{n})$
		\end{center}
	\end{thm} 
	\begin{itemize}
		\item How many IVs are sufficient to determine $ K\textsubscript{n} $?
		\item Determine probability that $ S\textsubscript{0}$, $S\textsubscript{1}$, $S\textsubscript{n} $ remain unchanged
		\item Probability of that: $(\frac{253}{256})\textsuperscript{252} = 0.0513 \approx 5\%$ (Question 9)
		\item What is a sufficient number of IVs in order to recover $K\textsubscript{3}$?
	\end{itemize}
\end{frame}

\begin{frame}[fragile]{RC4 Attack}{Probability of recovering $K\textsubscript{3}$}
	\begin{lstlisting}[language=Python]
		success_probability = 0.05
		#Win probability
		target_probability = 0.95
		num_trials = 1

		#Go through the IVs
		while True:
			cumulative_probability = 1 - binom.cdf(0, num_trials, success_probability)
			if cumulative_probability >= target_probability:
				break
			num_trials += 1
		return num_trials
	\end{lstlisting}
\end{frame}

\begin{frame}[fragile]{RC4 Attack}{Probability of recovering $K\textsubscript{3}$}
	\begin{itemize}
		\item How many IVs needed for 
		\item $50\% \rightarrow$ $14$
		\item $95\% \rightarrow$ $60$
		\item Hence, $60$ often regarded as sufficient for determing $K\textsubscript{3}$ (Question 7)
		\item Hier nochmal gucken, was die Wahrscheinlichkeit ist, solche IVs zu bekommen 
		\item Ich braeuchte laut meinem Code 5 Millionen lol
	\end{itemize}
\end{frame}

\begin{frame}[fragile]{RC4 Attack}{Probability of recovering $K\textsubscript{n}$}
	\begin{itemize}
		\item Same probability for recovering $K\textsubscript{4}$, $K\textsubscript{5}$,\dots
		\item If correct $IV$ is found
	\end{itemize}
		\bigskip
		\small
		\begin{tabular}{l|c|c|c|c|c|c|c|c|c}
		$i$ & 0 & 1 & 2 & 3 & 4 & 5 & 6 + V & 9 + V + K3 & 10 + V + K3 + K4\\
		\hline
		$S\textsubscript{i}$ & 4 & 0 & 6 + V & 9 + V + K3 & 10 + V + K3 + K4 & 5 & 2 & 3 & 1\\
		\end{tabular}
		\normalsize
\end{frame}

\begin{frame}[fragile]{RC4 Attack}{Probability of recovering $K\textsubscript{n}$}
	\small
	\begin{lstlisting}[language=Python]
		for x in range(plaintextlength):
			i = (i + 1) % 256
			j = (j + S[i]) % 256
			currentValue = S[i]
			S[i] = S[j]
			S[j] = currentValue
			t = (S[i] + S[j]) % 256
			keystream.append(S[t])
		return keystream
	\end{lstlisting}
	\normalsize

	\begin{itemize}
		\item $kB = S[t] = S[4] = 10 + V + K\textsubscript{3} + K\textsubscript{4}$
	\end{itemize}
\end{frame}

\begin{frame}[fragile]{RC4 Attack}{Probability of recovering $K\textsubscript{n}$}
	\begin{itemize}
		\item $\rightarrow$ Same probability for recovering $K\textsubscript{n}$
		\item Also doable with IVs of other form
		\item Suppose, $IV = (2,253,1)$ for recovering $K\textsubscript{3}$ (Question 10)
		\item Then, after $i=3$ steps, the $S-box$ will have the following form:
	\end{itemize}

	\begin{tabular}{l|c|c|c|c|c|c|r}
		$i$ & 0 & 1 & 2 & 3 & \dots & 3 + $K\textsubscript{3}$ & \dots \\
		\hline
		$S\textsubscript{i}$ & 0 & 2 & 1 & 3 +$K\textsubscript{3}$ & \dots & 3 & \dots\\
	\end{tabular}
\end{frame}

\begin{frame}[fragile]{RC4 Attack}{Determine useful IVs}
	\begin{thm}
		Let $kN$ be the keystreambyte at position $n$ we are looking for. We define\\
		\begin{center}$IV \dagger kN$,
		\end{center}
		if the given $IV$ is useful for the attacker to recover $kN$.
		To check if a given $IV=(x,y,z)$ is useful for the attack, we calculate the $s-box$ until step $i=n$ and apply:
		$S[i] + S[S[i]] \stackrel{?}{=} n \rightarrow IV \dagger kN$.
	\end{thm} 
\end{frame}

\section{Preventing attacks}

\begin{frame}[fragile]{Prevention against RC4 attacks}{Many improved algorithms}
	\begin{itemize}
		\item Performance $\leftrightarrow$ Security trade-off
		\item \textbf{RC4+}: Best security, but 3x execution time
		\begin{itemize}
			\item Uses three layers of scrambling the s-box 
		\end{itemize}
		\item \textbf{Improved RC4}: Improved security and parallel execution
		\begin{itemize}
			\item Focus on altering PRGA by adding $\oplus$ operations and using two S-boxes
		\end{itemize}
		\item \textbf{Effective RC4}: Faster and more secure
		\begin{itemize}
			\item Same KSA as Improved KSA 
			\item IN PRGA, two output bytes are produces and XORed with plaintext bytes
		\end{itemize}
		\item \textbf{RC4FMS}: Decreased chances of a succesful FMS attack
		\begin{itemize}
			\item Adds more randomness to the first 4 bytes 
		\end{itemize}
	\end{itemize}
\end{frame}

\begin{frame}[fragile]{Prevention against RC4 attacks}{Many improved algorithms}
	\begin{itemize}
		\item Add 256 \textbf{more steps} to the initialization process and discard them afterwards
		\item Use alternative protocols such as \textbf{WPA2/WPA3} with other encryption algorithms
		\item \textbf{Increase IV sizes} to at least $32$ bits $\rightarrow 2^{8}$ times for attacker to find collisions/useful IVs (Question 11)
		\item Use other hashing algorithms such as \textbf{MD5, SHA-1}
	\end{itemize}
\end{frame}

\begin{frame}[fragile]{RC4 summary}{Everything we have learnt}
	\begin{itemize}
		\item Invented in 1987 by Ron Rivest as stream cipher with \textbf{variable key length}
		\item Officially \textbf{outdated} because of too many weaknesses
		\item Consists of two parts
		\begin{itemize}
			\item \textbf{KSA}
			\item \textbf{PRGA}
		\end{itemize}
		\item Used in \textbf{WEP} and \tetxbf{SSL/TLS}, now replaced by other protocols/other encryption algorithms
		\item In-depth look at specific \textbf{FMS attack on RC4 in WEP}, makes use of weak IVs
		\item Numerous \textbf{improved RC4 variants} for better security, offer too many \textbf{trade-offs} compared to other algorithms
	\end{itemize}
\end{frame}
%\appendix

\makethankyou

\begin{frame}{References}
\bibliography
\end{frame}

\end{document}
