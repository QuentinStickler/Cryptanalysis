\documentclass[
	aspectratio=169,	% Modern aspect ratio (TODO: Other ratios not yet supported)
	onlytextwidth,		% Sets totalwidth=\textwidth and therefore e.g. columns won't invade the margins
	t,					% Default vertical alignment of frames and colums at top (default is centered) % Stored in \beamer@centered (\beamer@centeredfalse, \beamer@centeredtrue)
%	handout,			% Create a basic handout of the presentation (removes overlays)
	]{beamer}

%%%%%%%%%%%%%%%%%%%%%%%%%%%%%%%%%%%%%%%%%%
% 1) Load the desired presentation theme
\usetheme[
% Individual options to customize the presentation conforming to the corporate design
	cb,					% Change default faculty color set and predefined faculty values: hs or <empty> (default), inw, cb, me, sw, wi, inst (\renewcommand{\insertfacultyname}{Institutname} needed)
	language=english,	% Change language to english (default: ngerman), other languages are possible (see babel-package) but may need further adjustments 
	toc,				% Adds a ToC slide
%	sectionslide,		% Display separate section slides
%	subsectionslide,		% Display separate subsection slides containing section and subsection name
%	smallpagenumber,	% Reduces the size of the page number
%	nototalpages,		% Hides the total number of slide in footline
%	nofacultyicon,		% Hides the faculty icon on title page
	colormath=nottext,	% Enables coloring of math text: off or <empty>, full, nottext (default)
%	printhandout,		% Places two slides on a single a4 paper for printing the presentation (beamer class option 'handout' needed)
%	noframesubtitle,	% Disables frame subtitles alltogether and slightly increases frame height
% Additional style options not completely conforming to the corporate design
	titlepagedate,		% Shows the date on the titelpage
%	fancystyle,			% Enables some fancy styles, that are not part of the corporate design specifications (default: off)
	progressbar,		% Shows the progressbar in footline (run twice to update progressbar)
	]{hsmw} 

%%%%%%%%%%%%%%%%%%%%%%%%%%%%%%%%%%%%%%%%%%
% 2) Specify default fields for presentation and pdf document properties
% Set the title: \title{Long title everywhere} or \title[Short title for footline]{Long title for titlepage}
\title[RC4-Algorithmus]{RC4 Algorithm}
% Authors (separate multiple author names e.g. with \and for additional space): \author{author for everywhere} or \author[author for footline]{author for titlepage and thankyouslide}
\author[Quentin Stickler]{Quentin Stickler, B.Sc.}
% Institute (will be prefilled automatically, depending on chosen faculty theme option): \institute{institute for everywhere} or \institute[institute for footline and thankyouslide]{institute for titlepage}
%\institute[institute for footline and thankyouslide]{institute for titlepage}
% Date of presentation (\today or a fixed value): \date{date for everywhere} or \date[date for footline]{date for titlepage}
\date{\today} % 22. März 2022
% Impressum for thankyou slide (leave one empty if not wanted or needed):
\email{qstickle@hs-mittweida.de} % \email{E-Mail}
%\phone{+49 3727 58-1598} % \phone[Mobile Number]{Telephone Number}
%\office{Haus 8 | Richard Stücklen-Bau | Raum 8-207\newline Am Schwanenteich 6b | 09648 Mittweida} % \office{Office}
%\courseofstudies{Informatik (IF10w1-M)} % \courseofstudies{Course of Studies (Group Number)}
%\additional[Sidebar Text]{Main Text} % Additional information you may want to give (Department, Module, etc.) \additional[Sidebar Text]{Main Text}

%%%%%%%%%%%%%%%%%%%%%%%%%%%%%%%%%%%%%%%%%%
% 3) Use features
% Titlegraphic changes the title page to "wide" (default) if left empty or inserts given image (by file path) and scales to 6.2cm height otherwise:
\titlegraphic{} %\titlegraphic{figures/thankyou.jpg}

%%%%%%%%%%%%%%%%%%%%%%%%%%%%%%%%%%%%%%%%%%
% 4) Add bibliography
% Load the package and options you like, e.g. for recommended ieee style in combination of biblatex and biber:
%\usepackage[backend=biber, bibstyle=ieee, citestyle=numeric-comp, sorting=none, natbib=true, hyperref=true, dashed=false]{biblatex}
% Add your bibliography file(s)
%\addbibresource{literature.bib} 
% Dont forget to use \makebibliography where you want to put it, e.g. at the end of your presentation, after the "normal" slides:
% \appendix
% \makebibliography
% Compile with following command sequence to fully include bibliography: pdflatex, biber, pdflatex, pdflatex

%%%%%%%%%%%%%%%%%%%%%%%%%%%%%%%%%%%%%%%%%%
% 5) Existing macro usage examples

% \appendix is used as end marker for slides (slide numbers and progressbar)
% \makethankyou creates a thank you slide and is used as end marker for slides (progressbar)
% \makebibliography creates one or multiple bibliography slides

% For multiple speakers you can use \setcurrentspeaker{speaker name} or \setcurrentspeaker*{speaker name} prior to the frame
% To reset to the default author, use \resetcurrentspeaker{} or \resetcurrentspeaker*{}
% Starred versions of these macros prepend the word "speaker"

%%%%%%%%%%%%%%%%%%%%%%%%%%%%%%%%%%%%%%%%%%
% 6) Import additional packages you need, (re-)define macros and create a wonderful latex presentation

% You can remove this package - it is only needed for the dummy content
\usepackage{blindtext}
\usepackage{pythonhighlight}
\usepackage{amsmath}
\usepackage{setspace}

%%%%%%%%%%%%%%%%%%%%%%%%%%%%%%%%%%%%%%%%%%
\begin{document}

\begin{frame}[fragile]{The Rise and Fall of RC4}{Why it's not really used anymore}
	\begin{itemize}
		\item Stream cipher with variable key-size length
		\item Used to be most wiedely used stream cipher in Software applications
		\item Invented in 1987 by Ron Rivest for RSA security
		\item Kept secret but got leaked in 1994
		\item Easy to implement and quite fast
		\item ...but also very vulnerable
		
	\end{itemize}
\end{frame}



\begin{frame}[fragile]{RC4 Algorithm}{How does it work?}
	\begin{itemize}
		\item Consists of two parts
		\item Part 1: Initialization
		\item Part 2: Keystream Generator
		\item S-Box (Array) with length of 256
		\item Two 8-byte sized counters i and j
	\end{itemize}
\end{frame}

\begin{frame}[fragile]{RC4 Initialization}{Part One: Filling S-Box and T-Box}
	\begin{itemize}
		\item S-Box with length 256
		\item Counters i and j set to 0
		\item Linear filling of the S-Box from 0 to 255 (S[0] = 0, S[1] = 1\dots)
		\item Following loop will be run:

		\begin{python}
			for x in range(256):		###Initilaze S-Box and T-Box
			S[x] = x
			T[x] = asciikey[x % keylength]
		\end{python}
		
	\end{itemize}
\end{frame}


\begin{frame}{Initialization}{Example}
	\begin{itemize}
		\item Text = "TestText"
		\item Key = "TestKey"
		\item S-Box = [0, 1, 2, 3 \dots, 255]
		\item Initialization of T-Box:
		\begin{itemize}
			\item Keylength = 7
			\item Ascii-Text = 84 101 115 116 75 101 121
		\end{itemize}
		\medskip
		$\begin{array}{|ccccccc|}
		84 & 101 & 115 & 116 & 75 & 101 & 121 \\
		84 & 101 & 115 & 116 & 75 & 101 & 121 \\
		\dots & \dots & \dots & \dots & \dots & \dots & \dots \\
		\dots & \dots & \dots & 84 & 101 & 115 & 116 \\
		\end{array}$
	\end{itemize}
\end{frame}


\begin{frame}[fragile]{RC4 Initialization}{Part Two: Permutation}
	\begin{itemize}
		\item Permutate S-Box based on given key
		\item We always use modulo n = 256 because of the given length

		\begin{python}
			j = 0
			for i in range(256):
				j = (j + S[i] + T[i]) % 256
				currentvalue = S[i]
				S[i] = S[j]
				S[j] = currentvalue
		\end{python}
		\item At the end: (Pseudo-)randomly generated S-Box
		
	\end{itemize}
\end{frame}


\begin{frame}{Permutation Example}
	\begin{itemize}
		\item S-Box Initialization:
		\medskip
		$\begin{array}{|ccccccc|}
		\textcolor{green}{0} & 1 & 2 & 3 & 4 & 5 & 6 \\
		\dots & \dots & \dots & \dots & \dots & \dots & \dots \\
		249 & 250 & 251 & 252 & 253 & 254 & 255 \\
		\end{array}$
		\item i = \textcolor{green}{0}
		\item j = (\textcolor{blue} {j} + \textcolor{green}{S[i]} + \textcolor{red}{T[i]}) mod(256)
		\item j = (\textcolor{blue} {84} + \textcolor{green} {0} + \textcolor{red} {84}) mod(256) = 168 mod (256) = 168
		\item Swap S[i] (0) and S[j] (84)
		\item S[i] = 84, S[j] = 0
	\end{itemize}
\end{frame}

\begin{frame}{Permutation Example Cont'd}
	$\begin{array}{|ccccccc|}
		84 & \textcolor{green}{1} & 2 & 3 & 4 & 5 & 6 \\
		\dots & \dots & \dots & \dots & \dots & \dots & \dots \\
		80 & 81 & 82 & 83 & 0 & 85 & 86 \\
		\dots & \dots & \dots & \dots & \dots & \dots & \dots \\
		249 & 250 & 251 & 252 & 253 & 254 & 255 \\
	\end{array}$
	\begin{itemize}
		\item i = \textcolor{green}{1}
		\item j = (\textcolor{blue} {j} + \textcolor{green}{S[i]} + \textcolor{red}{T[i]}) mod(256)
		\item j = (\textcolor{blue} {186} + \textcolor{green} {1} + \textcolor{red} {101}) mod(256) = 288 mod (256) = 32
		\item Swap S[i] (1) and S[j] (186)
		\item S[i] = 186, S[j] = 1
	\end{itemize}
\end{frame}

\begin{frame}{Permutation Example Cont'd}
	$\begin{array}{|ccccccc|}
		84 & 186 & \textcolor{green}{2} & 3 & 4 & 5 & 6 \\
		\dots & \dots & \dots & \dots & \dots & \dots & \dots \\
		80 & 81 & 82 & 83 & 0 & 85 & 86 \\
		\dots & \dots & \dots & \dots & \dots & \dots & \dots \\
		249 & 250 & 251 & 252 & 253 & 254 & 255 \\
	\end{array}$
	\begin{itemize}
		\item i = \textcolor{green}{2}
		\item j = (\textcolor{blue} {j} + \textcolor{green}{S[i]}) + \textcolor{red}{T[i]} mod(256)
		\item j = (\textcolor{blue} {47} + \textcolor{green} {2} + \textcolor{red} {115}) mod(256) = 126 mod (256) = 126
		\item Swap S[i] (1) and S[j] (47)
		\item S[i] = 47, S[j] = 2
	\end{itemize}
\end{frame}

\begin{frame}{Permutation Example}{Final S-Box Form}
	\tiny
	\bigskip
	\centering
	$\begin{array}{|cccccccccccccccc|}
		84& 186& 47& 208& 12& 95& 222& 212& 71& 9& 26& 246& 103& 38& 28& 165 \\
		138 & 68 & 130 & 10 & 50 & 143 & 72 & 155 & 39 & 139 & 112 & 16 & 79 & 78 & 196 & 146 \\
		216 & 179 & 159 & 178 & 34 & 119 & 59 & 56 & 63 & 183 & 53 & 197 & 100 & 236 & 101 & 4 \\
		176 & 250 & 116 & 67 & 5 & 60 & 194 & 35 & 105 & 87 & 118 & 218 & 97 & 168 & 1 & 77 \\
		44 & 229 & 25 & 48 & 141 & 42 & 175 & 91 & 94 & 211 & 121 & 169 & 215 & 89 & 99 & 24 \\
		98 & 164 & 181 & 129 & 255 & 185 & 110 & 8 & 220 & 154 & 109 & 219 & 201 & 153 & 120 & 62 \\
		51 & 0 & 217 & 37 & 20 & 226 & 43 & 127 & 170 & 227 & 243 & 249 & 133 & 126 & 161 & 156 \\
		82 & 167 & 140 & 115 & 145 & 74 & 182 & 83 & 184 & 104 & 189 & 81 & 52 & 233 & 172 & 245 \\
		157 & 66 & 124 & 177 & 102 & 80 & 147 & 171 & 106 & 162 & 70 & 30 & 199 & 6 & 69 & 18 \\
		173 & 45 & 32 & 88 & 125 & 221 & 7 & 65 & 75 & 158 & 232 & 128 & 237 & 190 & 108 & 248 \\
		13 & 144 & 2 & 46 & 49 & 31 & 134 & 123 & 92 & 40 & 114 & 254 & 131 & 213 & 41 & 93 \\
		117 & 253 & 23 & 137 & 234 & 209 & 224 & 136 & 107 & 90 & 202 & 223 & 132 & 27 & 15 & 207 \\
		73 & 195 & 239 & 64 & 206 & 251 & 149 & 228 & 231 & 166 & 187 & 214 & 86 & 242 & 191 & 76 \\
		192 & 58 & 142 & 61 & 57 & 193 & 33 & 244 & 180 & 205 & 111 & 3 & 122 & 36 & 22 & 14 \\
		240 & 252 & 238 & 188 & 247 & 85 & 203 & 174 & 200 & 11 & 148 & 152 & 160 & 230 & 210 & 29 \\
		96 & 235 & 163 & 150 & 17 & 204 & 54 & 55 & 198 & 151 & 225 & 21 & 135 & 113 & 19 & 241 \\
	\end{array}$
	\normalsize
	\begin{itemize}
		\item Result = Permutated S-Box
		\item All numbers from 0-255 in "random" places
	\end{itemize}
\end{frame}

\begin{frame}[fragile]{Keystream Generator}
	\begin{itemize}
		\item Generate keystream depending on length of given plaintext
		\begin{python}
   			for x in range(plaintextlength):
				i = (i + 1) % 256
				j = (j + S[i]) % 256
				currentValue = S[i]
				S[i] = S[j]
				S[j] = currentValue
				t = (S[i] + S[j]) % 256
				keystream.append(S[t])
    		return keystream
		\end{python}
	\end{itemize}
\end{frame}

\begin{frame}[fragile]{Keystream Generator}{Example, i = 0}
	\begin{itemize}
		\item i = 0, j = 0
		\item i = (0 + 1) mod 256 = 1
		\item j = (0 + S[i]) mod 256 = 186 mod 256 = 186
		\item Swap S[i] (186) and S[j] (202)
		\item t = (S[i] + S[j]) mod 256 = 388 mod 256 = 132
		\item keystream = [132, ]
	\end{itemize}
\end{frame}

\begin{frame}[fragile]{Keystream Generator}{Example, i = 1}
	\begin{itemize}
		\item i = 1, j = 186
		\item i = (1 + 1) mod 256 = 2
		\item j = (186 + 47) mod 256 = 233 mod 256 = 233
		\item Swap S[i] (47) and S[j] (11)
		\item t = (47 + 11) mod 256 = 58 mod 256 = 58
		\item keystream = [132, 58, ]
	\end{itemize}
\end{frame}

\begin{frame}[fragile]{Keystream Generator}{Example, i = 2}
	\begin{itemize}
		\item i = 2, j = 233
		\item i = (2 + 1) mod 256 = 3
		\item j = (233 + 208) mod 256 = 451 mod 256 = 185
		\item Swap S[i] (208) and S[j] (90)
		\item t = (208 + 90) mod 256 = 298 mod 256 = 42
		\item keystream = [132, 58, 42, ....]
		\item Final keystream = [132, 58, 42, 7, 129, 233, 245, 149]
	\end{itemize}
\end{frame}

\begin{frame}[fragile]{Encryption}
	\begin{itemize}
		\item Plaintext XOR keystream
	\end{itemize}
\end{frame}

\begin{frame}[fragile]{Decryption}
	\begin{itemize}
		\item Ciphertext XOR keystream
	\end{itemize}
\end{frame}

	\section{Benutzerdefinierte Anpassungen}

	\section{FAQ}

	\appendix
	\makethankyou
%	\makebibliography

	\section{\appendixname}

\end{document}
