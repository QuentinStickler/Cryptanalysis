\documentclass[
	aspectratio=169,	% Modern aspect ratio (TODO: Other ratios not yet supported)
	onlytextwidth,		% Sets totalwidth=\textwidth and therefore e.g. columns won't invade the margins
	t					% Default vertical alignment of frames and colums at top (default is centered) % Stored in \beamer@centered (\beamer@centeredfalse, \beamer@centeredtrue)
%	handout,			% Create a basic handout of the presentation (removes overlays)
	]{beamer}

%%%%%%%%%%%%%%%%%%%%%%%%%%%%%%%%%%%%%%%%%%
% 1) Load the desired presentation theme
\usetheme[
% Individual options to customize the presentation conforming to the corporate design
	cb,					% Change default faculty color set and predefined faculty values: hs or <empty> (default), inw, cb, me, sw, wi, inst (\renewcommand{\insertfacultyname}{Institutname} needed)
	language=english,	% Change language to english (default: ngerman), other languages are possible (see babel-package) but may need further adjustments 
	toc,				% Adds a ToC slide
%	sectionslide,		% Display separate section slides
%	subsectionslide,		% Display separate subsection slides containing section and subsection name
%	smallpagenumber,	% Reduces the size of the page number
%	nototalpages,		% Hides the total number of slide in footline
%	nofacultyicon,		% Hides the faculty icon on title page
	colormath=nottext,	% Enables coloring of math text: off or <empty>, full, nottext (default)
%	printhandout,		% Places two slides on a single a4 paper for printing the presentation (beamer class option 'handout' needed)
%	noframesubtitle,	% Disables frame subtitles alltogether and slightly increases frame height
% Additional style options not completely conforming to the corporate design
	titlepagedate,		% Shows the date on the titelpage
%	fancystyle,			% Enables some fancy styles, that are not part of the corporate design specifications (default: off)
%	progressbar,		% Shows the progressbar in footline (run twice to update progressbar)
	]{hsmw} 

%%%%%%%%%%%%%%%%%%%%%%%%%%%%%%%%%%%%%%%%%%
% 2) Specify default fields for presentation and pdf document properties
% Set the title: \title{Long title everywhere} or \title[Short title for footline]{Long title for titlepage}
\title[RC4-Algorithm]{RC4-Algorithm}
% Authors (separate multiple author names e.g. with \and for additional space): \author{author for everywhere} or \author[author for footline]{author for titlepage and thankyouslide}
\author[Quentin Stickler]{Quentin Stickler, B.Sc.}
% Institute (will be prefilled automatically, depending on chosen faculty theme option): \institute{institute for everywhere} or \institute[institute for footline and thankyouslide]{institute for titlepage}
%\institute[institute for footline and thankyouslide]{institute for titlepage}
% Date of presentation (\today or a fixed value): \date{date for everywhere} or \date[date for footline]{date for titlepage}
\date{24. April 2024} % 22. März 2022
% Impressum for thankyou slide (leave one empty if not wanted or needed):
\email{qstickle@hs-mittweida.de} % \email{E-Mail}
%\phone{+49 3727 58-1598} % \phone[Mobile Number]{Telephone Number}
%\office{Haus 8 | Richard Stücklen-Bau | Raum 8-207\newline Am Schwanenteich 6b | 09648 Mittweida} % \office{Office}
%\courseofstudies{Informatik (IF10w1-M)} % \courseofstudies{Course of Studies (Group Number)}
%\additional[Sidebar Text]{Main Text} % Additional information you may want to give (Department, Module, etc.) \additional[Sidebar Text]{Main Text}

%%%%%%%%%%%%%%%%%%%%%%%%%%%%%%%%%%%%%%%%%%
% 3) Use features
% Titlegraphic changes the title page to "wide" (default) if left empty or inserts given image (by file path) and scales to 6.2cm height otherwise:
\titlegraphic{} %\titlegraphic{figures/thankyou.jpg}

%%%%%%%%%%%%%%%%%%%%%%%%%%%%%%%%%%%%%%%%%%
% 4) Add bibliography
% Load the package and options you like, e.g. for recommended ieee style in combination of biblatex and biber:
%\usepackage[backend=biber, bibstyle=ieee, citestyle=numeric-comp, sorting=none, natbib=true, hyperref=true, dashed=false]{biblatex}
% Add your bibliography file(s)
%\addbibresource{literature.bib} 
% Dont forget to use \makebibliography where you want to put it, e.g. at the end of your presentation, after the "normal" slides:
% \appendix
% \makebibliography
% Compile with following command sequence to fully include bibliography: pdflatex, biber, pdflatex, pdflatex

%%%%%%%%%%%%%%%%%%%%%%%%%%%%%%%%%%%%%%%%%%
% 5) Existing macro usage examples

% \appendix is used as end marker for slides (slide numbers and progressbar)
% \makethankyou creates a thank you slide and is used as end marker for slides (progressbar)
% \makebibliography creates one or multiple bibliography slides

% For multiple speakers you can use \setcurrentspeaker{speaker name} or \setcurrentspeaker*{speaker name} prior to the frame
% To reset to the default author, use \resetcurrentspeaker{} or \resetcurrentspeaker*{}
% Starred versions of these macros prepend the word "speaker"

%%%%%%%%%%%%%%%%%%%%%%%%%%%%%%%%%%%%%%%%%%
% 6) Import additional packages you need, (re-)define macros and create a wonderful latex presentation

% You can remove this package - it is only needed for the dummy content
\usepackage{blindtext}
\usepackage{pythonhighlight}
\usepackage{amsmath}
\usepackage{setspace}
\usepackage{listings}
\usepackage{xcolor}
\usepackage[natbib=true, backend=biber, style= numeric, useprefix=true]{biblatex}
\addbibresource{sources.bib}

\definecolor{codegreen}{rgb}{0,0.6,0}
\definecolor{codegray}{rgb}{0.5,0.5,0.5}
\definecolor{codepurple}{rgb}{0.58,0,0.82}
\definecolor{backcolor}{rgb}{0.95,0.95,0.92}

%\beamerdefaultoverlayspecification{<+->}

\lstdefinestyle{Pythonstyle}{
	language=Python,
	backgroundcolor=\color{backcolor},
	commentstyle=\color{codegreen},
	keywordstyle=\color{magenta},
	numberstyle=\tiny\color{codegray},
	stringstyle=\color{codepurple},
	basicstyle=\ttfamily\footnotesize,
	breakatwhitespace=false,
	breaklines=true,
	captionpos=b,
	keepspaces=true,
	numbers=left,
	numbersep=5pt,
	showspaces=false,
	showstringspaces=false,
	showtabs=false,
	showtabs=false,
	tabsize=2
}
\lstset{style=Pythonstyle}

\newtheorem{thm}{Theorem}

%%%%%%%%%%%%%%%%%%%%%%%%%%%%%%%%%%%%%%%%%%
\begin{document}
\section{General info}
%\begin{itemize}
\begin{frame}[fragile]{General info}{History\footnote[frame]{\cite{stallings2005rc4}}}
	\begin{itemize}
		\item Stream cipher with \textbf{variable} key-size length
		\item Most widely used stream cipher in software applications in the past
		\item Invented in 1987 by Ron Rivest
		\item Kept secret but got leaked in 1994
		\item \textbf{Easy} to implement and quite \textbf{fast} (Encryption up to 10x faster than DES \footnote[frame]{\cite{stovsic2012rc4}})
		\item Offers a \textbf{lot of weaknesses und vulnerabilities}
		\item Better alternatives have been invented
		\item Now only used in private projects due to its simplicity and performance
	\end{itemize}
\end{frame}

\section{RC4-Algorithm in detail}

\begin{frame}[fragile]{RC4-Algorithm}{How does it work?}
	\begin{itemize}
		\item Consists of two parts
		\begin{itemize}
			\item Part 1: Key Scheduling Algorithm \textbf{(KSA)}
			\item Part 2: Pseudo Random Number Generator Algorithm \textbf{(PRGA)}
		\end{itemize}
		\item $S-Box$ (Array) with length of $256$
		\item Two 8-byte sized counters $i$ and $j$
		\item State space thus: $(2^{8})^2 * 256! \approx 2^{1700}$ \footnote[frame]{\footnotesize Question 6}
	\end{itemize}
\end{frame}

\begin{frame}[fragile]{Initialization}{Part One: Filling S-Box and K-Box}
	\begin{itemize}
		\item Counters $i$ and $j$ set to $0$
		\item Linear filling of the S-Box from $0$ to $255$ ($S[0] = 0$, $S[1] = 1$\dots)
		\item Store key bytes in seperate $K-Box$
	\end{itemize}
	\begin{lstlisting}[language=Python]
		for x in range(256):
			sbox[x] = x
			kbox[x] = key[x % len(key)]
	\end{lstlisting}
\end{frame}


\begin{frame}{Initialization}{Example}
	\begin{itemize}
		\item Text = "TestText"
		\item Key = "TestKey"
		\item S-Box = $[0, 1, 2, 3 \dots, 255]$
		\item Initialization of K-Box:
		\begin{itemize}
			\item Keylength = 7
			\item Ascii-Text = $84$ $101$ $115$ $116$ $75$ $101$ $121$
		\end{itemize}
		\medskip
		$\begin{array}{|ccccccc|}
		84 & 101 & 115 & 116 & 75 & 101 & 121 \\
		84 & 101 & 115 & 116 & 75 & 101 & 121 \\
		\dots & \dots & \dots & \dots & \dots & \dots & \dots \\
		\dots & \dots & \dots & 84 & 101 & 115 & 116 \\
		\end{array}$
	\end{itemize}
\end{frame}


\begin{frame}[fragile]{Initialization}{Part Two: Permutation}
	\begin{itemize}
		\item Permutate S-Box based on given key
		\item We always use modulo $n = 256$ because of the given length
		\begin{lstlisting}[language=Python]
			j = 0
			for i in range(256):
				j = (j + sbox[i] + kbox[i]) % 256
				Swap(sbox[i], sbox[j])
			return sbox
		\end{lstlisting}
		\item At the end: (Pseudo-)randomly generated S-Box \footnote[frame]{\cite{woungang20192nd}}
	\end{itemize}
\end{frame}


\begin{frame}{Permutation Example}{Keystream: [\textcolor{red} {84}, $101$, $115$, $116$, $75$, $101$, $121$]}
	\medskip
	$\begin{array}{|ccccccc|}
		\textcolor{green}{0} & 1 & 2 & 3 & 4 & 5 & 6 \\
		\dots & \dots & \dots & \dots & \dots & \dots & \dots \\
		249 & 250 & 251 & 252 & 253 & 254 & 255 \\
	\end{array}$
	\medskip
	\begin{itemize}
		\item i = \textcolor{green}{0}, j = \textcolor{blue}{0}
		\item j = (\textcolor{blue} {j} + \textcolor{green}{S[i]} + \textcolor{red}{K[i]}) $\%$ (256)
		\item j = (\textcolor{blue} {0} + \textcolor{green} {0} + \textcolor{red} {84}) $\%$ (256) = 84 $\%$ (256) = $84$
		\item Swap S[i] (0) and S[j] (84)
		\item S[i] = $84$, S[j] = $0$
	\end{itemize}
\end{frame}

\begin{frame}{Permutation Example Continued}{Keystream: [$84$, \textcolor{red} {101}, $115$, $116$, $75$, $101$, $121$]}
	$\begin{array}{|ccccccc|}
		\textcolor{blue}{84} & \textcolor{green}{1} & 2 & 3 & 4 & 5 & 6 \\
		\dots & \dots & \dots & \dots & \dots & \dots & \dots \\
		80 & 81 & 82 & 83 & 0 & 85 & 86 \\
		\dots & \dots & \dots & \dots & \dots & \dots & \dots \\
		249 & 250 & 251 & 252 & 253 & 254 & 255 \\
	\end{array}$
	\begin{itemize}
		\item i = \textcolor{green}{1}, j = \textcolor{blue} {84}
		\item j = (\textcolor{blue} {j} + \textcolor{green}{S[i]} + \textcolor{red}{K[i]}) $\%$ (256)
		\item j = (\textcolor{blue} {84} + \textcolor{green} {1} + \textcolor{red} {101}) $\%$ (256) = 186 $\%$ (256) = $186$
		\item Swap S[i] (1) and S[j] (186)
		\item S[i] = $186$, S[j] = $1$
	\end{itemize}
\end{frame}

\begin{frame}{Permutation Example Continued}{Keystream: [$84$, $101$, \textcolor{red} {115}, $116$, $75$, $101$, $121$]}
	$\begin{array}{|ccccccc|}
		84 & \textcolor{blue} {186} & \textcolor{green}{2} & 3 & 4 & 5 & 6 \\
		\dots & \dots & \dots & \dots & \dots & \dots & \dots \\
		80 & 81 & 82 & 83 & 0 & 85 & 86 \\
		\dots & \dots & \dots & \dots & \dots & \dots & \dots \\
		184 & 185 & 1 & 187 & 188 & 189 & 190\\
		\dots & \dots & \dots & \dots & \dots & \dots & \dots \\
		249 & 250 & 251 & 252 & 253 & 254 & 255 \\
	\end{array}$
	\begin{itemize}
		\item i = \textcolor{green}{2}, j = \textcolor{blue}{186}
		\item j = (\textcolor{blue} {j} + \textcolor{green}{S[i]} + \textcolor{red}{K[i]}) $\%$ (256)
		\item j = (\textcolor{blue} {186} + \textcolor{green} {2} + \textcolor{red} {115}) $\%$ (256) = 303 $\%$ (256) = $47$
		\item Swap S[i] (2) and S[j] (47)
		\item S[i] = $47$, S[j] = $2$
	\end{itemize}
\end{frame}

\begin{frame}{Permutation Example}{Final S-Box Form}
	\tiny
	\smallskip
	\centering
	$\begin{array}{|cccccccccccccccc|}
		84& 186& 47& 208& 12& 95& 222& 212& 71& 9& 26& 246& 103& 38& 28& 165 \\
		138 & 68 & 130 & 10 & 50 & 143 & 72 & 155 & 39 & 139 & 112 & 16 & 79 & 78 & 196 & 146 \\
		216 & 179 & 159 & 178 & 34 & 119 & 59 & 56 & 63 & 183 & 53 & 197 & 100 & 236 & 101 & 4 \\
		176 & 250 & 116 & 67 & 5 & 60 & 194 & 35 & 105 & 87 & 118 & 218 & 97 & 168 & 1 & 77 \\
		44 & 229 & 25 & 48 & 141 & 42 & 175 & 91 & 94 & 211 & 121 & 169 & 215 & 89 & 99 & 24 \\
		98 & 164 & 181 & 129 & 255 & 185 & 110 & 8 & 220 & 154 & 109 & 219 & 201 & 153 & 120 & 62 \\
		51 & 0 & 217 & 37 & 20 & 226 & 43 & 127 & 170 & 227 & 243 & 249 & 133 & 126 & 161 & 156 \\
		82 & 167 & 140 & 115 & 145 & 74 & 182 & 83 & 184 & 104 & 189 & 81 & 52 & 233 & 172 & 245 \\
		157 & 66 & 124 & 177 & 102 & 80 & 147 & 171 & 106 & 162 & 70 & 30 & 199 & 6 & 69 & 18 \\
		173 & 45 & 32 & 88 & 125 & 221 & 7 & 65 & 75 & 158 & 232 & 128 & 237 & 190 & 108 & 248 \\
		13 & 144 & 2 & 46 & 49 & 31 & 134 & 123 & 92 & 40 & 114 & 254 & 131 & 213 & 41 & 93 \\
		117 & 253 & 23 & 137 & 234 & 209 & 224 & 136 & 107 & 90 & 202 & 223 & 132 & 27 & 15 & 207 \\
		73 & 195 & 239 & 64 & 206 & 251 & 149 & 228 & 231 & 166 & 187 & 214 & 86 & 242 & 191 & 76 \\
		192 & 58 & 142 & 61 & 57 & 193 & 33 & 244 & 180 & 205 & 111 & 3 & 122 & 36 & 22 & 14 \\
		240 & 252 & 238 & 188 & 247 & 85 & 203 & 174 & 200 & 11 & 148 & 152 & 160 & 230 & 210 & 29 \\
		96 & 235 & 163 & 150 & 17 & 204 & 54 & 55 & 198 & 151 & 225 & 21 & 135 & 113 & 19 & 241 \\
	\end{array}$
	\medskip
	\normalsize
	\begin{itemize}
		\item Result = Permutated S-Box
		\item All numbers from $0-255$ in "random" places
	\end{itemize}
\end{frame}

\begin{frame}[fragile]{Keystream Generator}{Python Code}
	\begin{itemize}
		\item Generate keystream depending on length of given plaintext
		\begin{lstlisting}[language=Python]
			keystream = []
			i = 0
			j = 0
			for x in range(len(text)):
				i = (1 + i) % 256
				j = (sbox[i] + j) % 256
				
				Swap(sbox[i], sbox[j])
				result = sbox[i] + sbox[j] % 256
				keystream.append(sbox[result])   
			return keystream   
		\end{lstlisting}
	\end{itemize}
\end{frame}

\begin{frame}[fragile]{Keystream Generator}{Example, S-box = [$84$, \textcolor{green}{186}, $47$, $208$, $\dots$]}
	\begin{itemize}
		\item i = \textcolor{magenta}{0}, j = \textcolor{blue}{0}
		\item i = (\textcolor{magenta}{0} + 1) $\%$ 256 = \textcolor{olive}{1}
		\item \textit{\textcolor{gray}{j = (j + sbox[i]) \% 256}}
		\item j = (\textcolor{blue}{0} + \textcolor{green}{186}) $\%$ 256 = 186 $\%$ 256 = \textcolor{red}{186}
	\end{itemize}
	\medskip
	\begin{itemize}
		\item Swap S[\textcolor{olive}{1}] (\textcolor{green}{186}) and S[\textcolor{red}{186}] ($202$)
		\item \textit{\textcolor{gray}{result = sbox[i] + sbox[j] \% 256}}
		\item t = ($202$ + \textcolor{green}{186}) $\%$ 256 = 388 $\%$ 256 = $132$
		\item S[$132$] = $102$
		\item Keystream = [$102$, ]
	\end{itemize}
\end{frame}

\begin{frame}[fragile]{Keystream Generator}{Example, S-box = [$84$, $186$, \textcolor{green}{47}, $208$, $\dots$]}
	\begin{itemize}
		\item i = \textcolor{magenta}{1}, j = \textcolor{blue}{186}
		\item i = (\textcolor{magenta}{1} + 1) $\%$ 256 = \textcolor{olive}{2}
		\item \textit{\textcolor{gray}{j = (j + sbox[i]) \% 256}}
		\item j = (\textcolor{blue}{186} + \textcolor{green}{47}) $\%$ 256 = 233 $\%$ 256 = \textcolor{red}{233}
	\end{itemize}
	\medskip
	\begin{itemize}
		\item Swap S[{\textcolor{olive}{2}}] (\textcolor{green}{47}) and S[\textcolor{red}{233}] ($11$)
		\item \textit{\textcolor{gray}{result = sbox[i] + sbox[j] \% 256}}
		\item t = (\textcolor{green}{47} + 11) $\%$ 256 = 58 $\%$ 256 = $58$
		\item S[$58$] = $118$
		\item Keystream = [$102$, $118$, ]
	\end{itemize}
\end{frame}

\begin{frame}[fragile]{Keystream Generator}{Example, S-box = [$84$, $186$, $47$, \textcolor{green}{208}, $\dots$]}
	\begin{itemize}
		\item i = \textcolor{magenta}{2}, j = \textcolor{blue}{233}
		\item i = (\textcolor{magenta}{2} + 1) $\%$ 256 = \textcolor{olive}{3}
		\item \textit{\textcolor{gray}{j = (j + sbox[i]) \% 256}}
		\item j = (\textcolor{blue}{233} + \textcolor{green}{208}) $\%$ 256 = 451 $\%$ 256 = \textcolor{red}{185}
	\end{itemize}
	\medskip
	\begin{itemize}
		\item Swap S[\textcolor{olive}{3}] (\textcolor{green}{208}) and S[\textcolor{red}{185}] ($90$)
		\item \textit{\textcolor{gray}{result = sbox[i] + sbox[j] \% 256}}
		\item t = (\textcolor{green}{208} + $90$) $\%$ 256 = 298 $\%$ 256 = $42$
		\item S[$42$] = $53$
		\item Keystream = [$102$, $118$, $53$, ....]
		\item Final keystream = [$102$, $118$, $53$, $212$, $66$, $47$, $204$, $221$]
	\end{itemize}
\end{frame}

\begin{frame}[fragile]{Encryption}{Bytewise XOR}
	\small
	\begin{itemize}
		\item Plaintext = "TestText" = {\color{purple}[84, 101, 115, 116, 84, 101, 120, 116]}
		\item Keystream = {\color{olive}[102, 118, 53, 212, 66, 47, 204, 221]}
		\item Plaintext $\oplus$ Keystream =
		\item "0X320X130X460XA00X160X4A0XB40XA9" = [$50$, $19$, $70$,$160$, $22$, $74$, $180$, $169$]
	\end{itemize}
	\normalsize
\end{frame}

\begin{frame}[fragile]{Decryption}{Bytewise XOR}
	\small
	\begin{itemize}
		\item Ciphertext = "0X320X130X460XA00X160X4A0XB40XA9" = \\{\color{purple}[50, 19, 70, 160, 22, 74, 180, 169]}
		\item Keystream = {\color{olive}[102, 118, 53, 212, 66, 47, 204, 221]} 
		\item Ciphertext $\oplus$ Keystream = "TestText"
	\end{itemize}
	\normalsize
\end{frame}

\begin{frame}[fragile]{Summary}{RC4-Algorithm}
	\begin{itemize}
		\item Split up into two parts
		\begin{itemize}
			\item Part 1: Key Scheduling Algorithm \textbf{(KSA)}
			\item Part 2: Pseudo Random Number Generator Algorithm \textbf{(PRGA)}
		\end{itemize}
		\item $S-Box$ (Array) with length of $256$
		\item Permutate S-box based on given key
		\item Create a keystream for $\oplus$ en-/decrypting texts bytewise
	\end{itemize}
\end{frame}

\section{Attacking RC4}

\begin{frame}[fragile]{WEP}{Short summary\footnote[frame]{\cite{wepproblems}}}
	\begin{itemize}
		\item Wired Equivalent Protocol
		\item Used in IEEE 802.11 for protecting LAN users against eavesdropping
		\item Encrypt wirelessly transmitted packets
		\item Key used for encryption consists of a long-term key / root key ($rk$) and an initialization vector $IV$
		\item $RC4Key = IV||rk$
		\item Different public $IVs$ per packet, 24-bit-sized; $IV=(X,Y,Z)$
		\item 40/104-bit-sized secret $rk$
	\end{itemize}
\end{frame}

\begin{frame}[fragile]{Security problems in WEP}{Outdated since 2004 \footnote[frame]{\cite{wepcracking}}}
	\begin{itemize}
		\item "Swiss Cheese" of protocols $\rightarrow$ lots of security vulnerabilities
		\item \textbf{Small key sizes}; only 64-bit and 128-bit encryption key sizes
		\item CRC-32 for detecting changes made to data
		\begin{itemize}
			\item Useful for detecting errors but \textbf{useless for cryptographic validation}
			\item Attacker can easily alter the data so that the validation check is getting verified
		\end{itemize}
		\item \textbf{Small $IV$ sizes} of 24-bit $\rightarrow 2^{24}$ possibilities ($<17 million$)
	\end{itemize}
\end{frame}

\begin{frame}[fragile]{Attacking RC4 in WEP}{Utilizing IVs}
	\begin{itemize}
		\item Small key sizes (40-bit $rk$ and 24-bit $IV$)
		\item $IV$ is sent clearly together with packets
		\item Make use of "weak IVs" to recover $rk$ byte for byte
		\item \textbf{FMS attack} by Fluhrer, Shamir and Mantin in 2000\footnote[frame]{\cite{fluhrer2001weaknesses}}
	\end{itemize}
\end{frame}

\begin{frame}[fragile]{FMS Attack}{General process}
	\begin{itemize}
		\item Attacker graps a lot of transfered data
		\item Goal $\rightarrow$ Recover $rk$ $\rightarrow$ Decrypt all the ciphertexts
		\item Tries to catch $IVs$ of specific forms 
		\item Example: $ IV = (3, N-1, V)$ , where $N-1 = 255, V \in {0,\dots,255}$
		\item RC4-key of form $(3, 255, V, K\textsubscript{3}, K\textsubscript{4},K\textsubscript{5}, \dots)$
		\item $K\textsubscript{3}, K\textsubscript{4},K\textsubscript{5},\dots$ are the first unknown keybytes
		\item Exploiting the initialization phase
	\end{itemize}
\end{frame}

\begin{frame}[fragile]{FMS Attack}{Example for $K\textsubscript{3}$}
	\begin{itemize}
		\item Suppose, attacker has recoverd $IV = (\textcolor{green}{3}, 255, V)$ 
		\item Used for recovering $K\textsubscript{\textcolor{green}{3}}$
		\item Study S-Box during the initialization phase
		\item First, S-Box is set to the identitity permutation
	\end{itemize}
	\medskip
	\begin{table}[h!]
		\begin{center}
		  \begin{tabular}{l|c|c|c|c|c|c|r}
			$i$ & 0 & 1 & 2 & 3 & 4 & 5 & \dots\\
			\hline
			$S\textsubscript{i}$ & 0 & 1 & 2 & 3 & 4 & 5 & \dots\\
		  \end{tabular}
		\end{center}
	\end{table}
\end{frame}

\begin{frame}[fragile]{FMS Attack}{Example for $K\textsubscript{3}$ with $RC4Key=(3,255,V,K\textsubscript{3},K\textsubscript{4},\dots)$}
	\begin{itemize}
		\item At the first step $i=0$, we compute the next $j$
		\item $j = j + S\textsubscript{i} + K\textsubscript{i} = 0 + 0 + 3$ $\% (256) = 3$
		\item Thus, the elements at position $S\textsubscript{i}$ and $S\textsubscript{j}$ are swapped
		\medskip
		\begin{table}[h!]
			\begin{center}
			  \begin{tabular}{l|c|c|c|c|c|c|c}
				$i$ & 0 & 1 & 2 & 3 & 4 & 5 & \dots\\
				\hline
				$S\textsubscript{i}$ & \textcolor{green}{3} & 1 & 2 & \textcolor{green}{0} & 4 & 5 & \dots\\
			  \end{tabular}
			\end{center}
		  \end{table}
		\item At the next step $i = 1$, we compute $j$ as
		\item $j = 3 + S\textsubscript{i} + K\textsubscript{i} = 3 + 1 + 255$ $\% (256) = 3$
	\end{itemize}
\end{frame}

\begin{frame}[fragile]{FMS Attack}{Example for $K\textsubscript{3}$ with $RC4Key=(3,255,V,K\textsubscript{3},K\textsubscript{4},\dots)$}

	\begin{table}[h!]
		\begin{center}
			\begin{tabular}{l|c|c|c|c|c|c|r}
			$i$ & 0 & 1 & 2 & 3 & 4 & 5 & \dots\\
			\hline
			$S\textsubscript{i}$ & 3 & \textcolor{green}{0} & 2 & \textcolor{green}{1} & 4 & 5 & \dots\\
			\end{tabular}
		\end{center}
	\end{table}

	\begin{itemize}
		\item At the next step $i = 2$, we compute $j$ as
		\item $j = 3 + S\textsubscript{2} + K\textsubscript{2} = 3 + 2 + V$ $\% (256) = 5+V$
	\end{itemize}

	\begin{table}[h!]
		\begin{center}
			\begin{tabular}{l|c|c|c|c|c|c|c|c|r}
			$i$ & 0 & 1 & 2 & 3 & 4 & 5 & \dots & 5 + V & \dots\\
			\hline
			$S\textsubscript{i}$ & 3 & 0 & \textcolor{green}{5 + V} & 1 & 4 & 5 & \dots & \textcolor{green}{2} & \dots\\
			\end{tabular}
		\end{center}
	\end{table}

\end{frame}

\begin{frame}[fragile]{FMS Attack}{Example for $K\textsubscript{3}$ with $RC4Key=(3,255,V,K\textsubscript{3},K\textsubscript{4},\dots)$}

	
	\begin{itemize}
		\item At the next step $i = 3$, we compute $j$ as
		\item $j = 5 + V + S\textsubscript{3} + K\textsubscript{3} = 5 + V + 1 + K\textsubscript{3}$ $\%(256) = 6 + V + K\textsubscript{3}$
	\end{itemize}
	
	\begin{table}[h!]
		\begin{center}
			\begin{tabular}{l|c|c|c|c|c|c|c|c|c|c|r}
			$i$ & 0 & 1 & 2 & 3 & 4 & 5 & \dots & 5 + V & \dots & 6 + V + K\textsubscript{3} & \dots\\
			\hline
			$S\textsubscript{i}$ & 3 & 0 & 5 + V & \textcolor{green}{6 + V + K\textsubscript{3}} & 4 & 5 & \dots & 2 & \dots & \textcolor{green}{1} & \dots\\
			\end{tabular}
		\end{center}
	\end{table}

	\begin{itemize}
		\item Suppose $S\textsubscript{0}, S\textsubscript{1}$ and $S\textsubscript{3}$ will remain unchanged until step $i = 255$
		\item Then, the first keystreambyte will be computed following the keystream generator algorithm
	\end{itemize}
\end{frame}

\begin{frame}[fragile]{FMS Attack}{Recover $K\textsubscript{3}$ with $RC4Key=(3,255,V,K\textsubscript{3},K\textsubscript{4},\dots)$}

	\begin{lstlisting}[language=Python]
		keystream = []
		i = 0
		j = 0
		for x in range(len(text)):
			i = (i + 1) % 256
			j = (sbox[i] + j) % 256
			Swap(sbox[i], sbox[j])
			result = sbox[i] + sbox[j] % 256
			keystream.append(sbox[result])   
		return keystream   
	\end{lstlisting}
	\begin{itemize}
		\item i = $1$, j = $0$
		\item $KB\textsubscript{3} = (6 + V + K\textsubscript{3})$ $\% (256)$
	\end{itemize}

\end{frame}

\begin{frame}[fragile]{FMS Attack}{Recover $K\textsubscript{3}$ Continued}

	\begin{itemize}
		\item $KB\textsubscript{3} = (6 + V + K\textsubscript{3})$ $\% (256)$
		\item Suppose, Trudy can guess or knows the first byte of the plaintext, she can determine $K\textsubscript{3}$ with:
		\item $\rightarrow$ $K\textsubscript{3}  = KB\textsubscript{3} - 6 - V$ $\% (256)$
	\end{itemize}
\end{frame}


\begin{frame}[fragile]{FMS Attack}{Example for $K\textsubscript{\textcolor{blue}{4}}$ }
	\begin{itemize}
		\item $IV = (\textcolor{blue}{4},255,V)$ for $K\textsubscript{\textcolor{blue}{4}}$ after $i=\textcolor{blue}{4}$ steps:
	\end{itemize}

	\small
	
	\begin{table}[h!]
		\begin{center}
			\begin{tabular}{l|c|c|c|c|c|c|c|c|c}
			$i$ & 0 & 1 & 2 & 3 & 4 & 5 & 6 + V & 9 + V + K3 & 10 + V + K3 + K4\\
			\hline
			$S\textsubscript{i}$ & 4 & 0 & 6 + V & 9 + V + K3 & \textcolor{green}{10 + V + K3 + K4} & 5 & 2 & 3 & 1\\
			\end{tabular}
		\end{center}
	\end{table}

	\begin{itemize}
		\item i = 0, j = 0
		\item i = i + 1 = 1
		\item j = (j + $S\textsubscript{i}$) = 0 + 0 = 0
		\item Swap $S\textsubscript{i}$ and $S\textsubscript{j}$
		\item t = ($S\textsubscript{i}$ + $S\textsubscript{j}$) = 0 + 4 = \textcolor{blue}{4}
		\item $KB\textsubscript{4}$ = $S\textsubscript{t}$ = $S\textsubscript{\textcolor{blue}{4}}$
	\end{itemize}

	\normalsize

\end{frame}

\begin{frame}[fragile]{FMS Attack}{Example for $K\textsubscript{\textcolor{red}{5}}$}
	\begin{itemize}
		\item $IV = (\textcolor{red}{5},255,V)$ for $K\textsubscript{\textcolor{red}{5}}$ after $i=\textcolor{red}{5}$ steps:
	\end{itemize}

	\small

	\begin{table}[h!]
		\begin{center}
			\begin{tabular}{l|c|c|c|c|c}
			$i$ & 0 & 1 & 2 & 3 & \textcolor{red}{5}\\
			\hline
			$S\textsubscript{i}$ & 5 & 0 & 7 + V & 10 + V + K3 + & 14 + V + K3 + K4\\
			\end{tabular}
		\end{center}
	\end{table}

	\begin{table}[h!]
		\begin{center}
			\begin{tabular}{c|c|c|c|c}
			\textcolor{red}{5} & 7 + V & 10 + V + K3 & 14 + V + K3 + K4 & 15 + V + K3 + K4 + K5\\
			\hline
			\textcolor{magenta}{15 + V + K3 + K4 + K5} & 2 & 3 & 4 & 5\\
			\end{tabular}
		\end{center}
	\end{table}

	\begin{itemize}
		\item i = 0, j = 0
		\item j = (j + $S\textsubscript{i}$) = 0 + 0 = 0
		\item t = ($S\textsubscript{i}$ + $S\textsubscript{j}$) = 0 + 5 = \textcolor{red}{5}
		\item $KB\textsubscript{5}$ = $S\textsubscript{t}$ = $S\textsubscript{\textcolor{red}{5}}$
	\end{itemize}

\end{frame}

\begin{frame}[fragile]{Recovery of unknwon bytes}{General approach} 
	\begin{definition}
		Let $K\textsubscript{n} $ be the unknown key byte at position $n$. Let $ IV\textsubscript{n} $ be a tuple of $ (n, N-1, V) $,
		where $ N = 256, V \in {{0,\dots,255}}, n \geq 3 $ and $ KB\textsubscript{n} $ the known keystream byte at position $n$. Then \footnote[frame]{\footnotesize Question 8}: \\
		\begin{center}
			$K\textsubscript{n} = KB\textsubscript{n} - \sum_{1}^{n}x - V - (\sum_{3}^{n-1}K\textsubscript{n})$
		\end{center}
	\end{definition} 
	\begin{itemize}
		\item How many $IVs$ are sufficient to determine $ K\textsubscript{n} $?
		\item Determine probability that $ S\textsubscript{0}$, $S\textsubscript{1}$, $S\textsubscript{n} $ remain unchanged
	\end{itemize}
\end{frame}

\begin{frame}[fragile]{FMS Attack}{Probability of recovering $K\textsubscript{n}$}
	\begin{definition}
		Let $K\textsubscript{n}$ be the unknown key byte at position $n$, $N=256$ and $p = N-(n+1)$. Then the probability that the values in the given S-box at position $S\textsubscript{0}$,
		$S\textsubscript{1}$ and $S\textsubscript{n}$ will remain unchanged for $p$ steps, equals:\\
		\begin{center}
			$(\frac{253}{N})\textsuperscript{p}$
		\end{center}
	\end{definition} 
	\begin{itemize}
		\item Probability for recovering $K\textsubscript{3}$: $(\frac{253}{256})\textsuperscript{252} = 0.0513 \approx 5\%$ \footnote[frame]{\footnotesize Question 9}
		\item What is a sufficient number of $IVs$ in order to recover $K\textsubscript{3}$?
	\end{itemize}
\end{frame}

\begin{frame}[fragile]{FMS Attack}{Probability of recovering $K\textsubscript{3}$}
	\begin{lstlisting}[language=Python]
		success_probability = 0.05
		#Win probability
		target_probability = 0.95
		num_trials = 1

		#Go through the IVs
		while True:
			cumulative_probability = 1 - binom.cdf(0, num_trials, success_probability)
			if cumulative_probability >= target_probability:
				break
			num_trials += 1
		return num_trials
	\end{lstlisting}
\end{frame}

\begin{frame}[fragile]{FMS Attack}{Probability of recovering $K\textsubscript{3}$}
	\begin{itemize}
		\item How many IVs needed for 
		\item $50\% \rightarrow$ $14$
		\item $95\% \rightarrow$ $60$
		\item Hence, $60$ often regarded as sufficient for determining $K\textsubscript{3}$ \footnote[frame]{\footnotesize Question 7}
	\end{itemize}
\end{frame}

\begin{frame}[fragile]{FMS Attack}{Probability of recovering $K\textsubscript{n}$}
	\begin{itemize}
		\item Probability for recovering $K\textsubscript{4}$: $(\frac{253}{256})\textsuperscript{251} = 0.0518$
		\item Probability for recovering $K\textsubscript{5}$: $(\frac{253}{256})\textsuperscript{250} = 0.0525$
		\item Chance gets higher as we move through the $S-Box$
	\end{itemize}
\end{frame}

\begin{frame}[fragile]{FMS Attack}{How to determine useful IVs}
	\begin{definition}
		Let $K\textsubscript{n}$ be the unknown key byte at position $n$. \\
		Let $IV\textsubscript{n}$ be a 24-bit sized tuple $(x,y,z)$, where $x,y,z \in {0,\dots, 255}.$ We define\\
		\begin{center}$IV\textsubscript{n} \dagger K\textsubscript{n}$,
		\end{center}
		if the given $IV\textsubscript{n}$ is useful to recover $K\textsubscript{n}$.
		To check if a given $IV\textsubscript{n}$ is useful for the attack, we permutate the S-box until step $i=n$ and calculate:
		\begin{center}$S[i] + S[S[i]] \stackrel{?}{=} n \rightarrow IV\textsubscript{n} \dagger K\textsubscript{n}$.
		\end{center}
	\end{definition} 
\end{frame}

\begin{frame}[fragile]{FMS Attack}{Logic of recovering $K\textsubscript{n}$}

	\begin{lstlisting}[language=Python]
		keystream = []
		i = 0
		j = 0
		for x in range(len(text)):
			i = (i + 1) % 256
			j = (sbox[i] + j) % 256
			Swap(sbox[i], sbox[j])
			result = sbox[i] + sbox[j] % 256
			keystream.append(sbox[result])   
		return keystream   
	\end{lstlisting}
	\begin{itemize}
		\item Thus, $IVs$ of other forms useful as well!
	\end{itemize}
\end{frame}

\begin{frame}[fragile]{Usefule $IV\textsubscript{3}$ example}{For $K\textsubscript{3}$ \footnote[frame]{\footnotesize Question 10}}
	\begin{itemize}
		$IV=(3,253,254)$ $\dagger$ $K\textsubscript{3}$ 
	\end{itemize}
\end{frame}

\section{Preventing attacks}

\begin{frame}[fragile]{Prevention against RC4 attacks}{Many improved algorithms\footnote[frame]{\cite{jindal2017optimization}}}
	\begin{itemize}
		\item Performance $\leftrightarrow$ Security trade-off
		\item \textbf{RC4+}: Best security, but 3x execution time
		\begin{itemize}
			\item Uses three layers of scrambling the s-box 
		\end{itemize}
		\item \textbf{Improved RC4}: Improved security and parallel execution
		\begin{itemize}
			\item Focus on altering PRGA by adding $\oplus$ operations and using two S-boxes
		\end{itemize}
		\item \textbf{Effective RC4}: Faster and more secure
		\begin{itemize}
			\item Same KSA as Improved KSA 
			\item IN PRGA, two output bytes are produces and XORed with plaintext bytes
		\end{itemize}
		\item \textbf{RC4FMS}: Decreased chances of a succesful FMS attack
		\begin{itemize}
			\item Adds more randomness to the first 4 bytes 
		\end{itemize}
	\end{itemize}
\end{frame}

\begin{frame}[fragile]{Increase WEP security}{With regrads to RC4\footnote[frame]{\cite{wepcracking}}}
	\begin{itemize}
		\item Add 256 \textbf{more steps} to the initialization process and discard them afterwards
		\item \textbf{Increase IV sizes} to at least $32$ bits $\rightarrow 2^{8}$ times for attacker to find collisions/useful IVs \footnote[frame]{\footnotesize Question 11}
		\item Use other hashing algorithms such as \textbf{MD5, SHA-1}\footnote[frame]{\cite{wepproblems}}
		\item Use alternative protocols such as \textbf{WPA2/WPA3} with other encryption algorithms
	\end{itemize}
\end{frame}

\begin{frame}[fragile]{RC4 summary}{Everything we have learnt\footnote[frame]{\cite{stamp2007applied}}}
	\begin{itemize}
		\item Invented in 1987 by Ron Rivest as stream cipher with \textbf{variable key length}
		\item Officially \textbf{outdated} because of too many weaknesses
		\item Consists of two parts
		\begin{itemize}
			\item \textbf{KSA}
			\item \textbf{PRGA}
		\end{itemize}
		\item Used in \textbf{WEP} and \textbf{SSL/TLS}, now replaced by other protocols/other encryption algorithms
		\item In-depth look at specific \textbf{FMS attack on RC4 in WEP}, makes use of weak $IVs$
		\item Numerous \textbf{improved RC4 variants} for better security, offer too many \textbf{trade-offs} compared to other algorithms
	\end{itemize}
\end{frame}

\appendix

\makethankyou

\begin{frame}[allowframebreaks]
	\printbibliography
\end{frame}

\end{document}
